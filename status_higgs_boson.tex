\documentclass[12pt]{article} 

\usepackage{amsmath}
\RequirePackage[numbers,compress]{natbib}
\bibliographystyle{jhep}

\begin{document}

\section{Introduction}


The main objective of the LHC is to probe the Electroweak Symmetry Breaking (EWSB) mechanism that generates the masses of the known elementary particles in the SM. The discovery of the Higgs boson in 2012 by the ATLAS~\cite{ATLAS:2012yve} and the CMS~\cite{CMS:2012qbp} collaborations and the subsequent studies of its properties with the full data set from Run 1, from 2009 to 2012, with a center-of-mass energy of 7 TeV and 8 TeV, provided the first opportunity to study this mechanism. The data collected during the LHC Run 2, from 2015 to 2018, with a higher center-of-mass energy of 13 TeV and more robust dataset, revealed the compatibility of the Higgs boson and its role within the Standard Model (SM)~\cite{Glashow:1961tr,PhysRevLett.19.1264,PhysRevD.2.1285}.

In the SM, the electroweak interactions are described by a gauge field theory invariant under the SU(2)$_L\times$ U(1)$_Y$ symmetry group. The mechanism of EWSB~\cite{PhysRevLett.13.321,PhysRev.145.1156} provides a general framework to preserve the structure of these gauge interactions at high energies along with the generation of the observed masses of the $W$ and $Z$ gauge bosons. The EWSB mechanism posits a self-interacting complex EW doublet scalar field, whose CP-even neutral component acquires a vacuum expectation value (vev) $v \equiv 246 \;\text{GeV}$, which sets the scale of the symmetry breaking. Three massless Goldstone bosons are generated and are absorbed to give masses to the $W$ and $Z$ gauge bosons. The remaining component of the complex doublet becomes the Higgs boson, a new (and thusfar unique) fundamental scalar particle. The masses of all fermions are also a consequence of EWSB since the Higgs doublet is postulated to couple to the fermions through Yukawa interactions.

The initial measurements during the LHC Run 1 were accessible mainly through production and decay channels related to the couplings of the Higgs boson to the vector gauge bosons (the mediators of the electroweak interactions, $W^\pm$, $Z$ and $\gamma$, as well as the gluons, $g$, mediators of the strong interactions). The outstanding performance of the LHC Run 2, made it possible for the ATLAS and CMS experiments to independently and unambiguously establish the couplings of the Higgs boson to the charged fermions of the third generation (the top quark, the bottom quark, and the tau).

In all observed production and decay modes measured so far, the rates and differential measurements are found to be consistent, within experimental and theoretical uncertainties, with the SM predictions. In high resolution decay channels, such as the ones with four leptons (electrons or muons) or diphoton final states, the mass of the Higgs boson has been measured at the permill precision level.

Nevertheless, several channels are still out of reach experimentally and the couplings of the Higgs boson to light fermions are yet to be explored. Moreover, within the current precision, a more complex sector with additional states is not ruled out, nor has it been established whether the Higgs boson is an elementary particle or whether it has an internal structure like any other scalar particles observed before it.

Without the Higgs boson, the calculability of the SM would have been spoiled. In particular, perturbative unitarity~\cite{PhysRevLett.30.1268,PhysRevD.10.1145,LlewellynSmith:1973yud,PhysRevD.16.1519} would be lost at high energies since the longitudinal $W/Z$ boson scattering amplitude would grow with the center-of-mass energy. In addition, the radiative corrections to the gauge boson self-energies would exhibit dangerous logarithmic divergences that would be difficult to reconcile with EW precision data. With the discovery of the Higgs boson, the SM is a spontaneously broken gauge theory and, as such, it could a priori be consistently extrapolated well above the masses of the $W$ and $Z$ bosons. Hence, formally there is no need for new physics at the EW scale. However, as the SM Higgs boson is a scalar particle, at the quantum level it has sensitivity to possible new physics scales. Quite generally, the Higgs boson mass is affected by the presence of heavy particles and receives quantum corrections which destabilise the weak scale barring a large fine tuning of unrelated parameters. This is known as the Higgs naturalness or hierarchy problem~\cite{PhysRevD.3.1818,tHooft:1980xss}. It has been the prime argument for expecting new physics right at the TeV scale. New theoretical paradigms have been imagined, such as a new fermion-boson symmetry called supersymmetry (SUSY)~\cite{WESS19741} (for recent reviews, see Refs.~\cite{Martin:1997ns,Allanchach:2019wrx}), or the existence of strong interactions at a scale of the order of a TeV from which the Higgs boson would emerge as a composite state~\cite{Georgi:1986im} (see Refs.~\cite{Bellazzini:2014yua,Panico:2015jxa,Csaki:2015hcd} for recent reviews). Alternatively, new agents stabilising the weak scale could also be light but elusive, like in models of neutral naturalness~\cite{Chacko:2005pe,Chacko:2005un,Craig:2015pha,Craig:2014aea}. Other more recent scenarios~\cite{Graham:2015cka,Espinosa:2015eda,Craig:2014aea}, instead, rely on the cosmological evolution of the Universe to drive the Higgs boson mass to a value much smaller than the cutoff of the theory and aim at alleviating the hierarchy problem without the need for TeV scale new physics, even thought there might still be interesting and spectacular signatures~\cite{Graham:2015cka,Espinosa:2015eda,Craig:2014aea,Flacke:2016szy}. Beyond the naturalness problem, extensions of the SM Higgs sector without other low-energy particles have been proposed, for example, to provide explanations for the fermion mass hierarchies, see e.g. Ref. ~\cite{Bauer:2015fxa,Bauer:2015kzy}, to account for the Dark Matter abundance, see e.g. Ref.~\cite{Barbieri:2006dq}, or to modify the properties of the electroweak phase transition~\cite{Morrissey:2012db}. Such models with additional scalars provide grounds to explore new Higgs boson signals in concrete and complete scenarios, with different types of coupling structure to fermions and gauge bosons.

The Higgs boson is anyway special and, in the eight years since its discovery, it became a powerful tool to explore the manifestations of the SM and to probe the physics landscape beyond it. It might offer direct insights on what comes beyond the weak scale through possible sizeable effects on the Higgs boson properties. The Higgs boson couplings, however, are observed to be in good agreement with their SM predictions. This, together with the strong bounds from precision electroweak and flavour data, leaves open the possibility that the Higgs boson may well be elementary, weakly coupled and solitary up to the Planck scale, rendering the EW vacuum potentially metastable~\cite{Degrassi:2012ry,Alekhin:2012py,Buttazzo:2013uya}.

After completion of the first two runs, the LHC has only gathered approximately 5\% of its projected full dataset. During the second long shut down currently underway, the LHC is undergoing important upgrades in order to prepare for its high luminosity phase. The foreseen larger datasets to be collected during Run 3 and ultimately during the High Luminosity LHC (HL-LHC), will enable yet more fundamental and challenging measurements to explore new physics.

This review is organised as follow. Section 11.2 is a theoretical review of the SM Higgs boson, its properties, production mechanisms and decay rates. In Section 11.3, the experimental measurements are described. In Section 11.4, the combination of the main Higgs boson production and decay channels is presented. In Section 11.5, measurements of the main quantum numbers and CP properties of the Higgs boson are reported and the bounds on its total width are discussed. In Section 11.6, a general theoretical framework to describe the deviations of the Higgs boson couplings from the SM predictions is introduced and the experimental measurements of these Higgs couplings is reviewed. Measurements of differential cross sections are outlined. Section 11.7 presents, in detail, some interesting models proposed for extensions of the SM Higgs sector, addressing the hierarchy problem or not, and considers their experimental signatures. Section 11.8 provides a short summary and a brief outlook.

\section{The Standard Model and the mechanism of electroweak symmetry breaking}

In the SM~\cite{Glashow:1961tr,PhysRevLett.19.1264,PhysRevD.2.1285}, electroweak symmetry breaking~\cite{PhysRevLett.13.321,PhysRev.145.1156} is responsible for generating mass for the W and Z gauge bosons rendering the weak interactions short ranged. The SM scalar potential reads:

\begin{equation}
    V (\phi) = m^2\phi^\dagger\phi + \lambda(\phi^\dagger\phi)^2
\end{equation}

with the Higgs field $\phi$ being a self-interacting SU(2)$_L$ complex doublet (four real degrees of freedom) with weak hypercharge $Y = 1$ (the hypercharge is normalised such that $Q = T_{3L} + Y /2$, $Q$ being the electric charge and $T_{3L}$ the eigenvalue of the diagonal generator of SU(2)$_L$):

\begin{equation}
    \phi = \frac{1}{\sqrt{2}}
    \begin{pmatrix}
        \sqrt{2}\phi^+ \\
        \phi^0 + ia^0
    \end{pmatrix},
\end{equation}

where $\phi^0$ and $a^0$ are the CP-even and CP-odd neutral components, and $\phi^+$ is the complex charged component of the Higgs doublet, respectively. $V(\phi)$ is the most general renormalizable scalar potential. If the quadratic term is negative, the neutral component of the scalar doublet acquires a non-zero vacuum expectation value (VEV)

\begin{equation}
    \langle\phi\rangle = \frac{1}{\sqrt{2}}
    \begin{pmatrix}
        0 \\ v
    \end{pmatrix}
\end{equation}

with $\phi^0 = H+\langle\phi^0\rangle$ and $\langle\phi^0\rangle \equiv v$, inducing the spontaneous breaking of the SM gauge symmetry SU(3)$_C\times$SU(2)$_L\times$U(1)$_Y$ into SU(3)$_C\times$U(1)$_\text{em}$. The global minimum of the theory defines the ground state, and spontaneous symmetry breaking implies that there is a (global and/or local) symmetry of the system that is not respected by the ground state. From the four generators of the SU(2)$_L\times$U(1)$_Y$ SM gauge group, three are spontaneously broken, implying that they lead to non-trivial transformations of the ground state and indicate the existence of three massless Goldstone bosons identified with three of the four Higgs field degrees of freedom. The Higgs field couples to the $W_\mu$ and $B_\mu$ gauge fields associated with the SU(2)$_L\times$U(1)$_Y$ local symmetry through the covariant derivative appearing in the kinetic term of the Higgs Lagrangian,

\begin{equation}
    \mathcal{L}_\text{Higgs} = (D_\mu\phi)^\dagger(D^\mu\phi) V(\phi)
\end{equation}

where $D_\mu\phi=(\partial + ig\sigma^aW^a_\mu/2 + ig'YB_\mu/2)\phi$, $g$ and $g'$ are the SU(2)$_L$ and U(1)$_Y$ gauge couplings, respectively, and $\sigma^a$, $a$ = 1,2,3 are the usual Pauli matrices. As a result, the neutral and the two charged massless Goldstone degrees of freedom mix with the gauge fields corresponding to the broken generators of SU(2)$_L$ and U(1)$_Y$ and become, in the unitary gauge, the longitudinal components of the $Z$ and $W$ physical gauge bosons, respectively. The $Z$ and $W$ gauge bosons acquire masses,

\begin{equation}
    m_W^2 = \frac{g^2v^2}{4}, \hspace{7pt} m_Z^2 = \frac{(g'^2+g^2)}{4}
\end{equation}

The fourth generator remains unbroken since it is the one associated to the conserved U(1)$\text{em}$ gauge symmetry, and its corresponding gauge field, the photon, remains massless. Similarly the eight color gauge bosons, the gluons, corresponding to the conserved SU(3)$_C$ gauge symmetry with 8 unbroken generators, also remain massless (though confined inside hadrons and mesons as the result of the asymptotic freedom behaviour of QCD). Hence, from the initial four degrees of freedom of the Higgs field, two are absorbed by the $W^\pm$ gauge bosons, one by the $Z$ gauge boson, and there is one remaining degree of freedom, $H$, that is the physical Higgs boson — a new scalar particle first imagined by P. Higgs~\cite{PhysRevLett.13.321,PhysRev.145.1156}. The Higgs boson is neutral under the electromagnetic interactions and transforms as a singlet under SU(3)$_C$ and hence does not couple at tree level to the massless photons and gluons.

The fermions of the SM acquire mass through renormalisable interactions between the Higgs field and the fermions: the Yukawa interactions,

\begin{equation}
    \mathcal{L}_\text{Yukawa} = -\hat{h}_{d_{ij}} \bar{q}_{L_i}\phi d_{R_j} \hat{h}_{u_{ij}} \bar{q}_{L_i}\tilde{\phi} u_{R_j} \hat{h}_{l_{ij}} \bar{l}_{L_i}\phi e_{R_j} + h.c.
\end{equation}

which respect the symmetries of the SM but generate fermion
masses once EWSB occurs. In the Lagrangian above, $\tilde{\phi} = i\sigma_2\phi^*$ and $q_L$ ($l_L$) and $u_R$, $d_R$ ($e_R$) are the quark (lepton) SU(2)$_L$ doublets and singlets, respectively, while in each term, $\hat{h}_{X_{ij}}$ is parametrised by a $3\times3$ matrix in family space. The mass term for neutrinos is omitted, but could be added in an analogous manner to the up-type quarks when right-handed neutrinos are supplementing the SM particle content (neutrinos can also acquire Majorana masses via non-renormalisable dimension-5 interactions with the Higgs field~\cite{Weinberg:1979sa}). Once the Higgs field acquires a VEV, and after rotation to the fermion mass eigenstate basis that also diagonalizes the Higgs-fermion interactions, $\hat{h}_{f_{ij}} \rightarrow h_{f_i} \delta_{ij}$ , all fermions acquire a mass given by $m_{f_i} = h_{f_{i}} v/\sqrt{2}$. The indices $i$,$j$ = 1,2,3 refer to the three families in the up-quark, downquark or charged lepton sectors. It should be noted that the EWSB mechanism provides no additional insight on possible underlying reasons for the large variety of masses of the fermions, often referred to as the flavor hierarchy. The fermion masses, accounting for a large number of the free parameters of the SM, are simply translated into Yukawa couplings.


\bibliography{referenceFile}

\end{document}