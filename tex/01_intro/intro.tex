\chapter{Introduction}

The universe, while overwhelmingly vast, is comprised of a curiously small number of elementary particles.
These particles and their strong, weak, and electromagnetic interactions with each other are accurately described by the Standard Model (SM).
A major shortcoming of the SM was its inability to predict the masses of these particles.

This dissertation presents a precision measurement of the Higgs boson mass and using LHC proton-proton collision data from Run 2  data set from  

The SM was not able to predict the masses of these particles until 1964 when the Brout-Englert-Higgs mechanism suggested that 
It wasn't until 1964 that the Brout-Englert-Higgs mechanism gave a self-consistent way to :
by breaking the electroweak gauge symmetry of the vacuum would give rise to non-zero masses of the weak gauge bosons.
This would yield a secondary effect too:
there should exist a fundamental scalar boson which is the quantum of the so-called ``Higgs field''.
On July 4th, 2012, this Higgs boson was discovered.

At first glance, the universe appears to be an overwhelmingly vast and complicated place.
However upon closer inspection, it is comprised of only a few different kinds of fundamental particles.
Particle physics has given rise to the Standard Model (SM) which mathematically describes these constituents and their interactions with each other.




The Standard Model (SM) is an impressively accurate mathematical theory which describes the fundamental particles of the universe and the rules for their possible interactions.
Problematically though, the SM predicts that all particles are massless.


Get to the Higgs boson.

Why is it important?
Knowing the mass of the Higgs boson 





% The Universe is comprised of just a few basic building blocks: the fundamental particles.

% By understanding the rules by which these particles interact, we can ;
% this is the aim of the Standard Model (SM) of Particle Physics. 

%  are described mathematically by the impressively accurate Standard Model (SM) of Particle Physics.

% Although there are four fundamental forces of nature, the SM takes into account three of the four forces in nature:
