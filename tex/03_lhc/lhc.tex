\chapter{THE LARGE HADRON COLLIDER}

\section{Motivation for the LHC}
% OUTLINE
- Study the fundamental constituents of matter and their interactions.
- TO test the theories that particle physics theorists contrive:
it is the back-and-forth experimentalists and their machines must , 
- The ol' fashioned way: smash things together.

\section{Overview of the LHC}
- World's largest and most energetic hadron collider.
- Collides pp, p-Pb, Pb-Pb.
- Circular collider (26.7 Km)
- Underground.
- Swiss-Franc border.
- Before the LHC, LEP was originally in the tunnel.
    - LEP ran from TODO--TODO.

\section{The Journey of a Proton at the LHC}
The path to particle physics discovery begins with a simple red tank of hydrogen gas (H$_2$), having a mass of only TODO\Kg.

Protons get injected into Series of smaller accelerators
    - The \emph{Linac4}---a linear particle accelerator---accelerates hydride ions (H$^-$) to 160\MeV which eventually make their way into the \emph{Proton Synchrotron Booster} (PSB).
    - In the PSB, each hydride ion has its electron pair completely stripped away, leaving only the bare proton TODO: mention electric field?
    - The protons then enter a series of circular accelerators, each machine feeding protons into the next while increasing the proton com energy by at least 1 order of magnitude.
    - The flow 
    - These protons are then accelerated to 2\GeV at which point they are injected into the \emph{Proton Synchrotron} (PS).
    - The PS then increases the proton energy to 26\GeV to be fed into the \emph{Super Proton Synchrotron} (SPS).
    - The penultimate step is for the SPS to further energize the protons to 450\GeV a accelerates protons to com energy
Finally the protons enter the LHC.

Protons are further accelerated to the maximum energy of 6.5\TeV using RF cavities to kick them.
- The protons would 
- 1232 dipole magnets made of copper-clad niobium-titanium are used to turn the proton beams.
- 392 quadrupole magnets compress the proton bunches to make them more linear.
- The cryogenics of the 96\tonne of superfluid helium-4

Finally the proton bunches approach a collision point.
- Bunch crossing (BX)
- Out of more than 40 million \pp collisions that could have occurred, protons are so small that a mere 50 collisions take place on average (\ie only 0.000\,1\%).
- Frequency of BX: considering that proton bunches are spaced 25\ns apart, this means 
Just as the PS feeds protons into the SPS, which feeds protons into the LHC, so too is it being considered for the LHC to feed a new project---the 100\Km Future Circular Collider.


SPECS
- Luminosity
- rates
- data

\section{High-Luminosity LHC}

Who built it?
- CERN
- intern'l collaboration.

Experiments around 







Located on the border between France and Switzerland, sandwiched between the scenic Jura mountains to the west and the sprawling city of Geneva (Genève) to the east, is CERN:
the European Organization for Nuclear Research (French: \emph{Conseil Européean pour la Recherche Nucléaire}).
This international collaboration is responsible for the construction and commissioning of the world's largest and most powerful particle accelerator, the Large Hadron Collider (LHC).
The completion of this world-renowned feat was only possible through the careful efforts of thousands of scientists, engineers, administrators, \etc from all over the world.
At the time of this writing, CERN is associated with at least 33 countries, each of which is considered either a Member State, an Associate Member State, or an Observer.

The circular LHC ring straddles the Franco-Swiss border, approximately 100\meter below the surface of the earth (Fig.~\ref{fig:lhc_and_boosters}, Left).
The ring itself has a circumference of 26.7\Km, making its inscribed area (56.7$\Km^{2}$) almost four times greater than the area of the neighboring city of Geneva (15.9$\Km^{2}$).
This machine is not only a particle accelerator but also a proton-proton (\pp) collider, sending one beam of protons travelling clockwise and the other beam counterclockwise around the ring. 
%%%%%%%%%%%%%%%%%%%%
\begin{figure}[pbth]
\centering
\includegraphics[height=7cm,keepaspectratio]{figures/lhc/lhc_drawn_on_map_withpoints.png}
% \includegraphics[width=0.49\textwidth,height=10cm,keepaspectratio]{figures/lhc/lhc_and_all_boosters1.jpg}
    \caption{
    (Left) The LHC ring (bigger ring) and the Super Proton Synchrotron (smaller ring) with the nearby town of Geneva for size comparison. 
    The four red stars indicate the \pp collision points. 
    (Right) CERN's accelerator complex.
    } 
    \label{fig:lhc_and_boosters}
\end{figure}
%%%%%%%%%%%%%%%%%%%%

Contrary to what some people may think, protons are not sent one by one at each other, hoping for a collision.
% they are simply too small to precisely aim at one another and hope for a collision.
Instead 100 billion protons are packed together into a ``proton bunch''.
A single proton bunch is about the size of a human hair ($\approx$50\mum wide and $\approx$10\cm long). 
The clockwise and counterclockwise rings are filled to a maximum of 2808 proton bunches, each one spaced 25\ns apart, and then sent to collide. 

It requires an incredibly strong magnetic field to turn the protons as they make their revolutions around the LHC. 
Recall that charged particles bend in a magnetic field, via the Lorentz force. 
Therefore, the LHC is equipped with 1232 dipole magnets distributed all along the length of the beam pipe to keep the proton bunches turning in the tunnel.
The cross section of such a dipole magnet is shown in Figure~\ref{fig:lhc_dipole_xs}.
Each dipole magnet is 14.3\meter long, weighs 35\tonne, cost nearly 500\,KCHF to produce, and has nearly 11\,700 amps of current running through it. 
Only with such massive currents is it possible to generate the appropriate magnetic field strength of 8\tesla to keep the protons turning. 
The magnetic field is maintained by titanium-niobium coils, which are kept under cryogenic conditions using liquid helium to achieve the necessary temperature of 1.9\kelvin to reach a superconducting state; 
this temperature is colder than that of outer space!
%%%%%%%%%%%%%%%%%%%%
\begin{figure}[pbth]
\centering
\includegraphics[width=15cm,height=10cm,keepaspectratio]{figures/lhc/lhc_dipole_xs.jpg}
    \caption{
    A cross section of one of the 1232 dipole magnets which span the entire length of the LHC tunnel.} 
    \label{fig:lhc_dipole_xs}
\end{figure}
%%%%%%%%%%%%%%%%%%%%

% proton is a {\it hadron} collider 
% It is fed by the Super Proton Synchrotron 
There are only four specific ``Points'' along the LHC where the proton bunches actually cross, as shown in Figure~\ref{fig:lhc_and_boosters}.
At each of these four points, there is a unique and gigantic particle detector to catch all the decay products from the \pp collisions. 

% LPS SPS
% into a beam pipe going clockwise and another 2800 bunches going around counterclockwise, 
As the two bunches are just about to cross one another, they are squeezed down using quadrupole magnets, focusing the beams more tightly, increasing their chance for tasty \pp collisions.
%from 50 $\mu$m
During such a bunch crossing (BX), amazingly most of the protons just pass right by one another; 
out of the possible 100 billion possible collisions that could have occurred, Figure~\ref{plt:pileup} shows that on average only 32 collisions occurred per BX in the LHC 2018 run, according to a particle detector called CMS, described in Chapter~\ref{ch:CMS_detector}.
It should be mentioned that the luminosity of the LHC is on the order of \LHigh. %$10^{34}$ Hz/cm$^2$.
%%%%%%%%%%%%%%%%%%%%
\begin{figure}[pbth]
\centering
\includegraphics[width=10cm,height=10cm,keepaspectratio]{figures/lhc/pileup_pp_2018.png}
    \caption{Histogram showing the distribution of the average number of \pp collisions per proton bunch crossing (pile up) which CMS recorded during the LHC 2018 run.} 
    \label{plt:pileup}
\end{figure}
%%%%%%%%%%%%%%%%%%%%

As the proton bunches whiz around the LHC, they are given ``kicks'' from radio-frequency (RF) cavities, which accelerate the protons to a max speed of 99.999996\%$c$.
At this speed, \emph{each proton} carries 6.5\TeV of energy, such that a single \pp collision contains a monstrous center-of-mass energy of $\sqrt{s} = 13\TeV$:
more than enough energy to create new particles like top quarks, Higgs bosons, and potentially BSM particles.
%%%%%%%%%%%%%%%%%%%%
\begin{figure}[pbth]
    \centering
    \includegraphics[width=10cm,height=10cm,keepaspectratio]{figures/lhc/proton_proton_quarksandgluons.jpg}
        \caption{
        Two protons can be smashed together at very high energies to have their partons interact and convert the high energies into new kinds of matter.} 
        \label{fig:pp_collision}
    \end{figure}
    %%%%%%%%%%%%%%%%%%%%
% The LHC has ushered in the era of ``TeV-scale physics'',
% exploring the 
% - which is about the same amount of energy of a really fat flying mosquito.
In order to ``see'' such interesting particles, one needs to detect the outgoing particles produced from \pp collisions;
one needs a dedicated \emph{particle detector}...
The Compact Muon Solenoid detector should do the trick.
% It takes a proton \~90 $\mu$s 
% to make a complete revolution around the LHC moving at such a speed.

% \chapter{The Large Hadron Collider}  % Automatically turned into all caps.
% \label{ch:lhc}

% Located on the border between France and Switzerland, sandwiched between the beautiful Jura mountains to the west and the sprawling city of Geneva (Genève) to the east, is CERN:
% the European Organization for Nuclear Research 
% (Conseil Européean pour la Recherche Nucléaire).
% CERN is an international collaboration of more than 23 member states and its ``family'' is steadily growing.
% This collaboration is responsible for the construction and commissioning of the world's largest and most powerful particle accelerator:
% the Large Hadron Collider (LHC).

% The LHC collides particles using the brightest beams and highest energies that humans have ever made.
% Bright beams meaning highest luminosity.

% Higgs boson produced every 1 billion collisions.

% Beginning in 2026 (FIXME), the LHC will undergo a ``Phase 2'' upgrade and become the High-Luminosity LHC.
% This upgrade will increase the collider's luminosity by 10 fold (FIXME) and is predicted to deliver SO much data 3000 fb?.
