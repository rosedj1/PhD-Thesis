\section{Background Estimation}
\label{sec:bkg_estim}

Processes which pass the event selection (REF EVENT SELECTION) but are not actually the signal process of interest (HZZ4L) are called background processes.
These background events spoil the purity of signal and introduce further uncertainty into the final Higgs boson mass measurement.
Therefore, it is a priority to properly model and reduce the number of background events.

There are two types of backgrounds present in the HZZ4L analysis:

\begin{itemize}
    \item irreducible background
    \item reducible background
\end{itemize}






\subsection{Irreducible Background}
\label{sec:bkg_irred}

Backgrounds which produce 4 tight leptons in the final state look totally indistinguishable from the signal process.
Therefore, there is no way to filter out such background processes; they are irreducible.
There are two irreducible background processes for the HZZ4L analysis:

\begin{enumerate}
    \item ggTOZZ (gluon-gluon fusion)
    \item qqbarTOZZ (quark-antiquark annihilation)
\end{enumerate}






\subsection{Reducible Background}
\label{sec:bkg_red}

Besides irreducible backgrounds, there are other background processes that produce leptons which accidentally pass the analysis event selection.
These leptons should have been rejected since they come from a non-signal process, but due to imperfect lepton reconstruction, these leptons are identified as signal leptons.
Using clever analysis techniques and more efficient detectors these background processes can be reduced, hence the term reducible backgrounds.

Reducible backgrounds include:

\begin{enumerate}
    \item Z+jets
    \item ttbar
    \item WZ
\end{enumerate}

