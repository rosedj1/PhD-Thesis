\section{Background Estimation}
\label{sec:bkg_estim}

Processes which pass the signal event selection (REF EVENT SELECTION) but are not actually the signal process of interest (HZZ4L) are called background processes.
These background events spoil the purity of the signal events and introduce further uncertainty into the final Higgs boson mass measurement.
Therefore, it is a priority to properly model and reduce the number of background events.

The two types of background processes present in the HZZ4L analysis are:

\begin{itemize}
    \item irreducible background
    \item reducible background
\end{itemize}

and will be explained in full detail in the subsequent sections.

\subsection{Irreducible Background}
\label{sec:bkg_irred}

The first kind of background process is that which produces two Z bosons which then decay into four prompt leptons.
These four leptons typically get reconstructed as leptons which pass tight selection.
Therefore, the event is tagged (incorrectly) as a 4-lepton signal event.
Since these processes cannot be distinguished from the signal process and cannot be reduced, they are called irreducible backgrounds.
The two irreducible backgrounds for the HZZ4L analysis are:

\begin{enumerate}
    \item ggTOZZ (gluon-gluon fusion)
    \item qqbarTOZZ (quark-antiquark annihilation)
\end{enumerate}

\subsection{Reducible Background}
\label{sec:bkg_red}

Besides irreducible backgrounds, there are other background processes that produce non-prompt leptons which are erroneously reconstructed as passing tight selection, due to detector imperfections.
These leptons should have been rejected since they come from a non-signal process, the imperfect lepton reconstruction, these leptons appear to be signal leptons.
Using careful event selection methods and more efficient detectors, these background processes can be reduced, hence the term reducible backgrounds.
Reducible backgrounds include:

\begin{enumerate}
    \item Z+jets
    \item ttbar
    \item WZ
    \item qqbarTOZZ, ggTOZZ
\end{enumerate}

Since reducible backgrounds are not eliminated entirely within the signal region, they must be modelled and estimated.

More concretely, reducible backgrounds have three main sources:

\begin{enumerate}
    \item misidentifying light-flavor hadrons (e.g., PIONS) as leptons,
    \item heavy-flavor hadrons which decay mid-flight into leptons,
    \item and asymmetric photon conversions into electrons.
\end{enumerate}

