\chapter{Higgs Boson Mass Measurement in the \hzzfourl Channel}
\section{Introduction}
- Higgs discovery:
    - EACH OF THESE IS A SENTENCE
    - in 2012 by CMS, ATLAS
    - Big deal in particle physics, a huge discovery.
    - So far it is one of a kind: fundamental scalar particle.
    - Thus, we should continue to study its properties.
- Higgs properties:
    - Portal to BSM (ref: search for dilepton mass resonance)
    - There are many results on Higgs properties: mass, spin, charge, decay processes, lifetime.
    - mass is of particular importance: depending on mH and mtop, the stability of the Universe.
- ALL previous mass measurements:
    - Run 1:
        - H->2gamma VALUE
        - H->ZZ->4L VALUE
    - Run 2:
        - H->2gamma VALUE
        - (2016) H->ZZ->4L VALUE
        - H->bb
        - H->mumu
        - H->WW 
- Why this thesis is important:
    - This thesis is focused on the precision measurement of mH in hZZ4L decay, using Full Run 2 data set from CMS.
    - Run 2 provides more data -> more precision on measurements of Higgs properties.
    - Can also use new techniques: VX constraint.
    - Predict mH for Run 3, will start soon summer 2022 and provide an approximate 300? /fb of L int.
    - In 2026(?), HLLHC provides even more data. ref snowmass paper.
- Structure of CH:
    - general overview, workflow, logic of mass measurement (ref Section).
        - Want to measure the Higgs boson mass, so need Higgs bosons.
        - Use HZZ4L because S/B ratio is huge: 2.
        - Must be careful to not grab all 4l events because other processes (backgrounds, ref SECTION) also make 4l.
            - So need to sort signal from background.
        - 


            - Gather data from CMS.
            - What kind? 4lepton final state.
                - Why 4l final state?
                  
        - Make predictions using simulated samples.
        - FINISH
    - Data sets




At the LHC, the Higgs boson is produced in only 1 out of every billion pp collisions.
Even if \PH is produced, it will decay into two \PZ only a small percentage of the time (2.3\%).
This percentage is typically expressed as a fraction, called the \emph{branching fraction} or \emph{branching ratio} (\br).
there is a small probability of only  that it will decay into two \PZ bosons (2.3\%).
Those \PZ bosons then have only a small probability 
Furthermore, the Higgs boson has a mean lifetime of only $1 \tentotheminus{23}\snd$,
so the boson itself will never live long enough to interact directly with the CMS Detector.
Therefore, the Higgs boson can only be detected by the daughter particles into which it decays.




SEEMS TO BE A GOOD INTRO. Should it be the intro for the entire thesis?
