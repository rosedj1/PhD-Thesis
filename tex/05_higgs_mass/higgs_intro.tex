\chapter{HIGGS BOSON MASS MEASUREMENT IN THE \texorpdfstring{\hzzfourl}{H to ZZ to 4l} CHANNEL}
% Need to use \texorpdfstring{} so it also shows in pdf bookmarks.
\section{Introduction}
The Higgs boson was discovered in 2012 by the CMS and ATLAS collaborations.
This was a momentous achievement in particle physics because the existence of the Higgs boson was required to complete the SM.
In fact, it is sometimes referred to as the ``missing puzzle piece'' of the SM.
The Higgs boson is one of a kind: it is the only fundamental scalar particle ever discovered so far.
The unique boson could be a portal to new physics (\emph{beyond Standard Model physics}, BSM), \eg, by decaying into BSM low-mass dilepton mass resonances (Chapter~\ref{ch:dilep_res}).
In order to be certain that the recently discovered Higgs boson is truly the same one predicted by the SM, it is necessary to compare its measured properties to the predicted values.

Some properties of the Higgs boson can be predicted by the SM, like 
    - There are many results on Higgs properties: spin, charge, decay processes, lifetime, mass.
    - The last of these is of particular importance: depending on mH and mtop, the stability of the Universe.

- ALL previous mass measurements:
    - Run 1:
        - H->2gamma VALUE
        - H->ZZ->4L VALUE
    - Run 2:
        - H->2gamma VALUE
        - (2016) H->ZZ->4L VALUE
        - H->bb
        - H->mumu
        - H->WW 

- Why this thesis is important:
    - This thesis describes the methodology and results of the best precision measurement of mH to date by using the hZZ4l decay and Full Run 2 data set from CMS.
    - Run 2 provides more data -> more precision on measurements of Higgs properties.
    - In addition to more HZZ4l events, this analysis provides new techniques, specifically the VX constraint.
    - Predict mH for Run 3, will start soon summer 2022 and provide an approximate 300? /fb of L int.
    - In 2026(?), HLLHC provides even more data. ref snowmass paper.

This chapter is structured as follows:
\begin{itemize}
    \item General overview of the ingredients of the Higgs boson mass measurement (Section~\ref{sec:analysis_overview}).
    \item Data sets, simulated samples, triggers (Section~\ref{sec:datasets}).
    \item Event reconstruction and selection (Section~\ref{sec:evt_sel}).
    \item Background Estimation (Irred. and Reduc. Backgrounds).
    \item Signal Modeling: kinematic Discriminant, per-event mass uncertainties, VXBS constraint, reference to ad hoc studies in appendix.
    % \item Observables: Four-Lepton Invariant Mass, Per-Event Mass Uncertainty, Matrix Element-Based Kinematic Discriminant) (Section~\ref{sec:observables})
    \item Systematic Uncertainties.
    \item Results.
    \item Summary.
\end{itemize}

SEEMS TO BE A GOOD INTRO. Should it be the intro for the entire thesis?
