\chapter{HIGGS BOSON MASS MEASUREMENT IN THE \texorpdfstring{\hzzfourl}{H to ZZ to 4l} CHANNEL}
% Need to use \texorpdfstring{} so it also shows in pdf bookmarks.
\section{Introduction}
- Higgs discovery:
    - higgs discovered in 2012 by CMS, ATLAS collaborations was a momentous event in particle physics.
    - The Higgs boson is sometimes referred to as the ``missing puzzle piece'' of the SM.
    - It is the only fundamental scalar particle ever discovered to date which makes it truly one of a kind.
    - The Higgs boson could be a portal to new physics (\emph{Beyond Standard Model Physics}), \eg, decaying into a low-mass dilepton mass resonance (Chapter~\ref{ch:dilep_res}).
    - To be certain it is the predicted Higgs boson from the SM, it is necessary to measure its properties and compare them to the predictions of the SM.

- Higgs properties:
    - There are many results on Higgs properties: spin, charge, decay processes, lifetime, mass.
    - The last of these is of particular importance: depending on mH and mtop, the stability of the Universe.

- ALL previous mass measurements:
    - Run 1:
        - H->2gamma VALUE
        - H->ZZ->4L VALUE
    - Run 2:
        - H->2gamma VALUE
        - (2016) H->ZZ->4L VALUE
        - H->bb
        - H->mumu
        - H->WW 

- Why this thesis is important:
    - This thesis describes the methodology and results of the best precision measurement of mH to date by using the hZZ4l decay and Full Run 2 data set from CMS.
    - Run 2 provides more data -> more precision on measurements of Higgs properties.
    - In addition to more HZZ4l events, this analysis provides new techniques, specifically the VX constraint.
    - Predict mH for Run 3, will start soon summer 2022 and provide an approximate 300? /fb of L int.
    - In 2026(?), HLLHC provides even more data. ref snowmass paper.

This chapter is structured as follows:
\begin{itemize}
    \item General overview of the ingredients of the Higgs boson mass measurement (Section~\ref{sec:analysis_overview}).
    \item Data sets and simulated samples (Section~\ref{sec:datasets}).
    \item Object reconstruction (Section~\ref{sec:datasets})
    \item Event selection (Section~\ref{sec:datasets})
\end{itemize}

SEEMS TO BE A GOOD INTRO. Should it be the intro for the entire thesis?
