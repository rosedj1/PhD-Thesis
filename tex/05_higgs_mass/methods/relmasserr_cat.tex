% Expected mH measurement uncertainties using 3D pdf(m4l, sigma, Dkin | mH)
% with background (no systematics yet)
%\section{$D_{m_{4\ell}}$ categorization}
\subsection{Relative mass error categorization}
\label{sec:SignalParam_N_2D}
In the latest public results TODO:CITE, a 3D likelihood has been used to extract final Higgs boson mass uncertainty. 
For the full Run 2 results, a categorization based on the relative mass error is implemented.
This method will help not only in taking into account the correlation with the \Dkinbkg (see \ref{sec:DkinCorrelation}), but it will also improve the signal parameterization.
In particular, it has been found that the signal parameters of the DSCB not only depend on the mass but also on the relative mass error.\\
Starting from the raw distribution of the relative mass error, taken from ggH 125 \GeV sample, 
9 bins are defined guaranteeing an equal amount of raw events in each bin. Bin split 
is done independently for each year and for each final state. Example of the binning is given, 
for 2018, in  Fig.~\ref{Bin_splitting_2018}. First studies on categorizations can be found in TODO:CITE.
% \cite{HIG_18_002}.
\begin{figure}[!htbp]
\begin{center}
	\includegraphics[width=0.48\textwidth]{../../higgsmassmeasurement/AN-19-248/Figures/Bin_splitting/RelMassErroSplitting_4mu.pdf}
	\includegraphics[width=0.48\textwidth]{../../higgsmassmeasurement/AN-19-248/Figures/Bin_splitting/RelMassErroSplitting_4e.pdf}
  	\includegraphics[width=0.48\textwidth]{../../higgsmassmeasurement/AN-19-248/Figures/Bin_splitting/RelMassErroSplitting_2e2mu.pdf}
	\includegraphics[width=0.48\textwidth]{../../higgsmassmeasurement/AN-19-248/Figures/Bin_splitting/RelMassErroSplitting_2mu2e.pdf}
\caption{
	Relative mass error distribution for 2018 ggH sample @ 125 GeV: black lines show 
	the bins used for the categorization.}
\label{Bin_splitting_2018}
\end{center}
\end{figure}
To extract mass measurement using categorization, the signal is modelled independently in each bin, 
obtaining final measurements combining data cards. \\
Examples of the signal parameterization for 2018 are shown in 
Fig.~\ref{signal_lineshape_2018_4mu}(4$\mu$),
 \ref{signal_lineshape_2018_4e} (4e), 
 \ref{signal_lineshape_2018_2e2mu} (2e2$\mu$) and \ref{signal_lineshape_2018_2mu2e} (2$\mu$2e). 
 Each figure shows the fit of the 125 \GeV ggH sample (in blue, top left) and
the simultaneous fits of the all five mass points (red scale) in few random bins.
%Fig.~\ref{signal_lineshape_2016}(2$\mu$2e 2016),
% \ref{signal_lineshape_2017} (2e2$\mu$ 2017), 
% \ref{signal_lineshape_2018_1} (4e 2018) and  \ref{signal_lineshape_2018_2} (4$\mu$ 2018). 
%for 125 GeV sample, and in  Fig.~\ref{signal_lineshape_2016_full_1}, 
%\ref{signal_lineshape_2016_full_2}, \ref{signal_lineshape_2017_full_1}, 
%\ref{signal_lineshape_2017_full_2}, \ref{signal_lineshape_2018_full_1} and  
%\ref{signal_lineshape_2018_full_2}, for the simultaneous fits.
\begin{figure}[!htbp]
\begin{center}
	\includegraphics[width=0.48\textwidth]{../../higgsmassmeasurement/AN-19-248/Figures/Categorisation/DSCB_4mu_ggF_2018_2.pdf}
	\includegraphics[width=0.48\textwidth]{../../higgsmassmeasurement/AN-19-248/Figures/Categorisation/DSCB_4mu_ggF_2018_3.pdf}
	\includegraphics[width=0.48\textwidth]{../../higgsmassmeasurement/AN-19-248/Figures/Categorisation/DSCB_4mu_ggF_2018_8.pdf}
	\includegraphics[width=0.48\textwidth]{../../higgsmassmeasurement/AN-19-248/Figures/Categorisation/DSCB_4mu_ggF_2018_9.pdf}
\caption{
Fit examples for four $D'^{VXBS}_{m_{4\ell}}$ bins (2nd, 3rd, 8th, and 9th) 
for 4$\mu$ 2018. Top left: 125 GeV sample only fit. The others plots stand for the simultaneous 
fit of all the mass points: starting from top right respectively 125, 120, 124, 126, 130 GeV.}
\label{signal_lineshape_2018_4mu}
\end{center}
\end{figure}

\begin{figure}[!htbp]
\begin{center}
	\includegraphics[width=0.48\textwidth]{../../higgsmassmeasurement/AN-19-248/Figures/Categorisation/DSCB_4e_ggF_2018_1.pdf}
	\includegraphics[width=0.48\textwidth]{../../higgsmassmeasurement/AN-19-248/Figures/Categorisation/DSCB_4e_ggF_2018_2.pdf}
	\includegraphics[width=0.48\textwidth]{../../higgsmassmeasurement/AN-19-248/Figures/Categorisation/DSCB_4e_ggF_2018_5.pdf}
	\includegraphics[width=0.48\textwidth]{../../higgsmassmeasurement/AN-19-248/Figures/Categorisation/DSCB_4e_ggF_2018_9.pdf}
\caption{
Fit examples for four $D'^{VXBS}_{m_{4\ell}}$ bins (1st, 2nd, 5th, and 9th) 
for 4e 2018. Top left: 125 GeV sample only fit. The others plots stand for the simultaneous 
fit of all the mass points: starting from top right respectively 125, 120, 124, 126, 130 GeV.}
\label{signal_lineshape_2018_4e}
\end{center}
\end{figure}

\begin{figure}[!htbp]
\begin{center}
	\includegraphics[width=0.48\textwidth]{../../higgsmassmeasurement/AN-19-248/Figures/Categorisation/DSCB_2e2mu_ggF_2018_1.pdf}
	\includegraphics[width=0.48\textwidth]{../../higgsmassmeasurement/AN-19-248/Figures/Categorisation/DSCB_2e2mu_ggF_2018_3.pdf}
	\includegraphics[width=0.48\textwidth]{../../higgsmassmeasurement/AN-19-248/Figures/Categorisation/DSCB_2e2mu_ggF_2018_7.pdf}
	\includegraphics[width=0.48\textwidth]{../../higgsmassmeasurement/AN-19-248/Figures/Categorisation/DSCB_2e2mu_ggF_2018_9.pdf}
\caption{
Fit examples for four $D'^{VXBS}_{m_{4\ell}}$ bins (1st, 3rd, 7th, and 9th) 
for 2e2$\mu$ 2018. Top left: 125 GeV sample only fit. The others plots stand for the simultaneous 
fit of all the mass points: starting from top right respectively 125, 120, 124, 126, 130 GeV.}
\label{signal_lineshape_2018_2e2mu}
\end{center}
\end{figure}

\begin{figure}[!htbp]
\begin{center}
	\includegraphics[width=0.48\textwidth]{../../higgsmassmeasurement/AN-19-248/Figures/Categorisation/DSCB_2mu2e_ggF_2018_1.pdf}
	\includegraphics[width=0.48\textwidth]{../../higgsmassmeasurement/AN-19-248/Figures/Categorisation/DSCB_2mu2e_ggF_2018_4.pdf}
	\includegraphics[width=0.48\textwidth]{../../higgsmassmeasurement/AN-19-248/Figures/Categorisation/DSCB_2mu2e_ggF_2018_6.pdf}
	\includegraphics[width=0.48\textwidth]{../../higgsmassmeasurement/AN-19-248/Figures/Categorisation/DSCB_2mu2e_ggF_2018_9.pdf}
\caption{
Fit examples for four $D'^{VXBS}_{m_{4\ell}}$ bins (1st, 4th, 6th, and 9th) 
for 2$\mu$2e 2018. Top left: 125 GeV sample only fit. The others plots stand for the simultaneous 
fit of all the mass points: starting from top right respectively 125, 120, 124, 126, 130 GeV.}
\label{signal_lineshape_2018_2mu2e}
\end{center}
\end{figure}
Looking in each bin at the parameter mean of the DSCB function, used to fit the 125 \GeV sample, 
(Fig.~\ref{MeanDependence}), it can be seen that if in 4$\mu$ final state (black dots) the mean
is quite stable, in the 4e (in red) and 2e2$\mu$ (green) final states, it shifts towards lower values
($\mathcal{O}$({GeV})). Categorization will help in properly describe the Higgs boson line shape
not only as a function of mass but also as a function of mass resolution.
% \footnote{As we went 
% through this exercise, we also uncovered a minor bug in how $f(m_{4\ell}, \sigma_{m_{4\ell}} | m_H)$ 
% was built in the past; the bug affected $4e$ and, 
% to a lesser degree, $2e2\mu$ final states.}

%This could be seen in Fig.~\ref{Mass_1D_vs_1Derr} where a direct comparison of the 
%1D mass distribution is compared with the as distribution obtained after the convolution with the 
%relative mass error. Looking in particular at 4e and 2e2$\mu$ final state, the two distributions
%show differences; on the opposite, in the case of 4$\mu$ and 24$\mu$2e final states, 
%the two distributions are quite similar. 
\begin{figure}[!htbp]
\begin{center}
		\includegraphics[width=0.48\textwidth]{../../higgsmassmeasurement/AN-19-248/Figures/Categorisation/MeanDependenceFromSigma_20160.pdf}
		\includegraphics[width=0.48\textwidth]{../../higgsmassmeasurement/AN-19-248/Figures/Categorisation/MeanDependenceFromSigma_20165.pdf}
		\includegraphics[width=0.48\textwidth]{../../higgsmassmeasurement/AN-19-248/Figures/Categorisation/MeanDependenceFromSigma_2017.pdf}
		\includegraphics[width=0.48\textwidth]{../../higgsmassmeasurement/AN-19-248/Figures/Categorisation/MeanDependenceFromSigma_2018.pdf}
\caption{
DSCB mean values as a function of mass resolution for the four different final states: 4$\mu$ (black),
4e (red), 2e2$\mu$ (green) and 2$\mu$2e (blue). Each bin on the x axis stands for a bin 
in categorization (moving from left to right, the relative mass error comes bigger). 
Top 2016 (left pre-VFP, right post-VFP), bottom left 2017, bottom right 2018.}
\label{MeanDependence}
\end{center}
\end{figure}
Finally, examples of the fits for normalisation can be observed in 
Fig.~\ref{signal_normalization_20160} (2016 pre-VFP ggH),
Fig.~\ref{signal_normalization_20165} (2016 post-VFP ggH), 
\ref{signal_normalization_2017} (2017 ggH) and 
\ref{signal_normalization_2018} (2018 ggH)
\begin{figure}[!htbp]
\begin{center}
		\includegraphics[width=0.96\textwidth]{../../higgsmassmeasurement/AN-19-248/Figures/Categorisation/Yield_ggF_20160.pdf}
\caption{
Normalization fit for ggH 2016 pre-VFP, for different decay channels, as a function
of mass, for the 9 bins of $D'^{VXBS}_{m_{4\ell}}$.}
\label{signal_normalization_20160}
\end{center}
\end{figure}
\begin{figure}[!htbp]
\begin{center}
		\includegraphics[width=0.96\textwidth]{../../higgsmassmeasurement/AN-19-248/Figures/Categorisation/Yield_ggF_20165.pdf}
\caption{
Normalization fit for ggH 2016 post-VFP, for different decay channels, as a function
of mass, for the 9 bins of $D'^{VXBS}_{m_{4\ell}}$.}
\label{signal_normalization_20165}
\end{center}
\end{figure}
\begin{figure}[!htbp]
\begin{center}
		\includegraphics[width=0.96\textwidth]{../../higgsmassmeasurement/AN-19-248/Figures/Categorisation/Yield_ggF_2017.pdf}
\caption{
Normalization fit for ggH 2017, for different decay channels, as a function
of mass, for the 9 bins of $D'^{VXBS}_{m_{4\ell}}$.}
\label{signal_normalization_2017}
\end{center}
\end{figure}
\begin{figure}[!htbp]
\begin{center}
		\includegraphics[width=0.96\textwidth]{../../higgsmassmeasurement/AN-19-248/Figures/Categorisation/Yield_ggF_2018.pdf}
\caption{
Normalization fit for ggH 2018, for different decay channels, as a function
of mass, for the 9 bins of $D'^{VXBS}_{m_{4\ell}}$.}
\label{signal_normalization_2018}
\end{center}
\end{figure}

%In Fig.~\ref{Mass_1D_vs_1Derr_categorization} it can be seen the comparison of 1D and 1Derr
%mass distribution for different final state but now in several \textbf{mass error} bins: now, the
%agreement is quite good in all final states.
%\begin{figure}[!htbp]
%\begin{center}
%		\includegraphics[width=0.35\textwidth]{../../higgsmassmeasurement/AN-19-248/Figures/Placeholder.png}
%\caption{
%Comparison of the mass distribution from 1D (green) and 1Derr (black)
%models in different \textbf{mass error} bins for .... 
%}
%\label{Mass_1D_vs_1Derr_categorization}
%\end{center}
%\end{figure}

\subsubsection{Result using categorization in 1D model}
%The mass error uncertainty evaluated in \ref{sec:EBE} is combined with the four-lepton mass to built
%a two-dimentional likelihood function, $\mathcal{L}$($m_{4\ell}$, \massUnc$|m_{H}$), where again 
%$m_{H}$ is fixed to the value of 125 GeV. \\
%\tablename~\ref{table:2D_model_result_year} shows inclusive results compared with 1D result
%for each year.
The expected $\mass{H}$ measurement uncertainty, in case of no-bkg and no-syst,
implementing the categorization, is reported in \tablename~\ref{table:2D_model_result}
split for different final state and in \tablename~\ref{table:2D_model_result_year} for different
years.
\begin{table}[ht]	
\begin{center}
    \topcaption{Expected Higgs boson mass uncertainty measured implementing the categorization,
    for different final state.
    All mass values are given in \MeV.  
    Statistical only results are considered at this stage of the analysis.
    }
    \begin{tabular}{ccccccc}
    \hline			
    Expected uncertainty	&	4$\mu$	&	4e	&	2e2$\mu$	&2$\mu$2e	& inclusive & Rel. Improvement \\
    \hline			
        N-1$D'_{VXBS}$ (No bkg)	&	124	&	319	&	203	&	236	&	92	&	-6\%	\\
        1$D'_{VXBS}$ (No bkg)	&	129	&	357	&	223	&	255	&	99	&	-8\%	\\
    %	N-1$D'_{VXBS}$ (No bkg)	&	128	&	371	&	227	&	253	&	98	&	-7\%	\\
    %	1$D'_{VXBS}$ (No bkg)	&	134	&	424	&	250	&	279	&	105	&	-	\\
    \hline
    %relative improvement	&	-	&	-	&	-	&	-	&	-	\\
    %\hline
    \end{tabular}
\label{table:2D_model_result}
\end{center}
\end{table}
\begin{table}[ht]	
\begin{center}
    \caption{Expected Higgs boson mass uncertainty measured implementing the categorization,
    for different years.
    All mass values are given in \MeV.  
    Statistical only results are considered at this stage of the analysis.
    }
    \begin{tabular}{ccccc} % TODO: Add total uncertainty in final column on right-hand side?
    \hline			
    Expected uncertainty	&	2016 pre-VFP	&	2016 post-VFP	&	2017	&	2018	\\
    \hline			
        N-1$D'_{VXBS}$ (no-bkg)	&	247	&	255	&	170	&	140	\\
        1$D'_{VXBS}$ (No bkg)	&	266	&	276	&	182	&	148	\\
    %	N-1$D'_{VXBS}$ (no-bkg)	&	192	&	180	&	147	\\
    %	1$D'_{VXBS}$ (No bkg)	&	206	&	194	&	157	\\
    \hline
    \end{tabular}
    \label{table:2D_model_result_year}
\end{center}
\end{table}
