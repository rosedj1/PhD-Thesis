% Expected mH measurement uncertainties using 3D pdf(m4l, sigma, Dkin | mH)
% with background (no systematics yet)
%\section{$D_{m_{4\ell}}$ categorization}
\subsection{Relative Mass Error Categorization}
\label{sec:SignalParam_N_2D}
Previously, a 3D likelihood fit has been used to extract final Higgs boson mass uncertainty. 
For the full Run 2 results, a categorization based on the relative mass error is implemented.
This method will help, not only in taking into account its correlation with \Dkinbkg (see \ref{sec:DkinCorrelation}), but it will also improve the signal parameterization.
In particular, the signal parameters of the DSCB depend on both the mass and the relative mass error.

Starting from the raw distribution of the relative mass error and a simulated ggH sample with $\mH \sim 125\GeV$, 
the distribution is divided into 9 equal-entry bins (\ie in order to guarantee an equal number of raw events in each bin).
Bin spliting is done separately for each year and for each final state.
The procedure is demonstrated in \cref{Bin_splitting_2018} for 2018 MC.
\begin{multiFigure}
    \centering
        \addFigure{0.48}{../../higgsmassmeasurement/AN-19-248/Figures/Bin_splitting/RelMassErroSplitting_4mu.pdf}
        \addFigure{0.48}{../../higgsmassmeasurement/AN-19-248/Figures/Bin_splitting/RelMassErroSplitting_4e.pdf}
        \addFigure{0.48}{../../higgsmassmeasurement/AN-19-248/Figures/Bin_splitting/RelMassErroSplitting_2e2mu.pdf}
        \addFigure{0.48}{../../higgsmassmeasurement/AN-19-248/Figures/Bin_splitting/RelMassErroSplitting_2mu2e.pdf}
    \captionof{figure}
        [Distribution of the relative mass error for a 2018 ggH sample, where $\mH \sim 125\GeV$]
        {Distribution of the relative mass error for a 2018 ggH sample, where $\mH \sim 125\GeV$.
        The dotted black lines show the equal-entry bin edges used for the categorization.
        \;A) \fourmu final state.
        \;B) \foure final state.
        \;C) \twoetwomu final state.
        \;D) \twomutwoe final state.}
    \label{Bin_splitting_2018}
\end{multiFigure}

To extract the mass measurement using this categorization method,
the signal is modeled independently in each bin, 
yielding individual (semi)final measurements which are then combined in the data cards.

Examples of the signal parameterization for 2018 samples are shown in \cref{signal_lineshape_2018_4mu,signal_lineshape_2018_4e,signal_lineshape_2018_2e2mu,signal_lineshape_2018_2mu2e}. 
Each figure shows the fit of the 125\GeV ggH sample (in blue, top left) and the simultaneous fits of the all five mass points (red scale) in few random bins.
% TODO: Reword
\begin{multiFigure}
    \centering
        \addFigure{0.48}{../../higgsmassmeasurement/AN-19-248/Figures/Categorisation/DSCB_4mu_ggF_2018_2.pdf}
        \addFigure{0.48}{../../higgsmassmeasurement/AN-19-248/Figures/Categorisation/DSCB_4mu_ggF_2018_3.pdf}
        \addFigure{0.48}{../../higgsmassmeasurement/AN-19-248/Figures/Categorisation/DSCB_4mu_ggF_2018_8.pdf}
        \addFigure{0.48}{../../higgsmassmeasurement/AN-19-248/Figures/Categorisation/DSCB_4mu_ggF_2018_9.pdf}
    \captionof{figure}
        [Fit examples for four $D'^{\text{VXBS}}_{m_{4\ell}}$ bins (2nd, 3rd, 8th, and 9th) for \fourmu 2018]
        {Fit examples for four $D'^{\text{VXBS}}_{m_{4\ell}}$ bins (2nd, 3rd, 8th, and 9th) for \fourmu 2018.
        Top left: 125\GeV sample only fit. The others plots stand for the simultaneous fit of all the mass points:
        starting from top right respectively 125, 120, 124, 126, 130\GeV.
        \;A) 2nd bin.
        \;B) 3rd bin.
        \;C) 8th bin.
        \;D) 9th bin.}
\label{signal_lineshape_2018_4mu}
\end{multiFigure}
% TODO: Reword
\begin{multiFigure}
    \centering
        \addFigure{0.48}{../../higgsmassmeasurement/AN-19-248/Figures/Categorisation/DSCB_4e_ggF_2018_1.pdf}
        \addFigure{0.48}{../../higgsmassmeasurement/AN-19-248/Figures/Categorisation/DSCB_4e_ggF_2018_2.pdf}
        \addFigure{0.48}{../../higgsmassmeasurement/AN-19-248/Figures/Categorisation/DSCB_4e_ggF_2018_5.pdf}
        \addFigure{0.48}{../../higgsmassmeasurement/AN-19-248/Figures/Categorisation/DSCB_4e_ggF_2018_9.pdf}
    \captionof{figure}
        [Fit examples for four $D'^{\text{VXBS}}_{m_{4\ell}}$ bins (1st, 2nd, 5th, and 9th) for \foure 2018]
        {Fit examples for four $D'^{\text{VXBS}}_{m_{4\ell}}$ bins (1st, 2nd, 5th, and 9th) 
        for \foure 2018. Top left: 125 GeV sample only fit. The others plots stand for the simultaneous 
        fit of all the mass points: starting from top right respectively 125, 120, 124, 126, 130\GeV.
        \;A) 1st bin.
        \;B) 2nd bin.
        \;C) 5th bin.
        \;D) 9th bin.}
    \label{signal_lineshape_2018_4e}
\end{multiFigure}
% TODO: Reword
\begin{multiFigure}
    \centering
        \addFigure{0.48}{../../higgsmassmeasurement/AN-19-248/Figures/Categorisation/DSCB_2e2mu_ggF_2018_1.pdf}
        \addFigure{0.48}{../../higgsmassmeasurement/AN-19-248/Figures/Categorisation/DSCB_2e2mu_ggF_2018_3.pdf}
        \addFigure{0.48}{../../higgsmassmeasurement/AN-19-248/Figures/Categorisation/DSCB_2e2mu_ggF_2018_7.pdf}
        \addFigure{0.48}{../../higgsmassmeasurement/AN-19-248/Figures/Categorisation/DSCB_2e2mu_ggF_2018_9.pdf}
    \captionof{figure}
        [Fit examples for four $D'^{\text{VXBS}}_{m_{4\ell}}$ bins (1st, 3rd, 7th, and 9th) for \twoetwomu 2018]
        {Fit examples for four $D'^{\text{VXBS}}_{m_{4\ell}}$ bins (1st, 3rd, 7th, and 9th) 
        for \twoetwomu 2018. Top left: 125 GeV sample only fit. The others plots stand for the simultaneous 
        fit of all the mass points: starting from top right respectively 125, 120, 124, 126, 130 GeV.
        \;A) 1st bin.
        \;B) 3rd bin.
        \;C) 7th bin.
        \;D) 9th bin.}
        \label{signal_lineshape_2018_2e2mu}
\end{multiFigure}
% TODO: Reword
\begin{multiFigure}
    \centering
        \addFigure{0.48}{../../higgsmassmeasurement/AN-19-248/Figures/Categorisation/DSCB_2mu2e_ggF_2018_1.pdf}
        \addFigure{0.48}{../../higgsmassmeasurement/AN-19-248/Figures/Categorisation/DSCB_2mu2e_ggF_2018_4.pdf}
        \addFigure{0.48}{../../higgsmassmeasurement/AN-19-248/Figures/Categorisation/DSCB_2mu2e_ggF_2018_6.pdf}
        \addFigure{0.48}{../../higgsmassmeasurement/AN-19-248/Figures/Categorisation/DSCB_2mu2e_ggF_2018_9.pdf}
    \captionof{figure}
        [Fit examples for four $D'^{\text{VXBS}}_{m_{4\ell}}$ bins (1st, 4th, 6th, and 9th) 
        for \twomutwoe 2018]
        {
            % Fit examples for four $D'^{\text{VXBS}}_{m_{4\ell}}$ bins (1st, 4th, 6th, and 9th) 
        for \twomutwoe 2018. Top left: 125 GeV sample only fit. The others plots stand for the simultaneous 
        fit of all the mass points: starting from top right respectively 125, 120, 124, 126, 130 GeV.
        \;A) 1st bin.
        \;B) 4th bin.
        \;C) 6th bin.
        \;D) 9th bin.} % TODO: double check bin numbers.
    \label{signal_lineshape_2018_2mu2e}
\end{multiFigure}
Looking in each bin at the parameter mean of the DSCB function, used to fit the 125\GeV sample, 
(\cref{MeanDependence}), it can be seen that if in \fourmu final state (black dots) the mean
is quite stable, in the \foure (in red) and \twoetwomu (green) final states, it shifts towards lower values
($\mathcal{O}$({GeV})).
Categorization  helps to properly describe the Higgs boson line shape, not only as a function of mass, but also as a function of mass resolution.
% TODO: Reword
\begin{multiFigure}
    \centering
        \addFigure{0.48}{../../higgsmassmeasurement/AN-19-248/Figures/Categorisation/MeanDependenceFromSigma_20160.pdf}
        \addFigure{0.48}{../../higgsmassmeasurement/AN-19-248/Figures/Categorisation/MeanDependenceFromSigma_20165.pdf}
        \addFigure{0.48}{../../higgsmassmeasurement/AN-19-248/Figures/Categorisation/MeanDependenceFromSigma_2017.pdf}
        \addFigure{0.48}{../../higgsmassmeasurement/AN-19-248/Figures/Categorisation/MeanDependenceFromSigma_2018.pdf}
    \captionof{figure}
        [DSCB mean values as a function of mass resolution for the four different final states]
        {DSCB mean values as a function of mass resolution for the four different final states: \fourmu (black),
        \foure (red), \twoetwomu (green) and \twomutwoe (blue). Each bin on the x axis stands for a bin 
        in categorization (moving from left to right, the relative mass error comes bigger). 
        \;A) 2016 pre-VFP.
        \;B) 2016 post-VFP.
        \;C) 2017 pre-VFP.
        \;D) 2018 pre-VFP.}
    \label{MeanDependence}
\end{multiFigure}
Finally, examples of the fits for normalisation can be observed in 
Fig.~\ref{signal_normalization_20160} (2016 pre-VFP ggH),
Fig.~\ref{signal_normalization_20165} (2016 post-VFP ggH), 
\ref{signal_normalization_2017} (2017 ggH) and 
\ref{signal_normalization_2018} (2018 ggH)
\begin{multiFigure}
    \centering
        \includegraphics[width=0.96\textwidth]{../../higgsmassmeasurement/AN-19-248/Figures/Categorisation/Yield_ggF_20160.pdf}
    \captionof{figure}
        [Normalization fit for ggH 2016 pre-VFP]
        {Normalization fit for ggH 2016 pre-VFP, for different decay channels, as a function of mass, for the 9 bins of
        % $D'^{\text{VXBS}}_{m_{4\ell}}$. % TODO
        }
    \label{signal_normalization_20160}
\end{multiFigure}
\begin{multiFigure}
    \centering
        \includegraphics[width=0.96\textwidth]{../../higgsmassmeasurement/AN-19-248/Figures/Categorisation/Yield_ggF_20165.pdf}
    \captionof{figure}
        [Normalization fit for ggH 2016 post-VFP]
        {Normalization fit for ggH 2016 post-VFP, for different decay channels, as a function of mass, for the 9 bins of $D'^{\text{VXBS}}_{m_{4\ell}}$.}
    \label{signal_normalization_20165}
\end{multiFigure}
\begin{multiFigure}
    \centering
		\includegraphics[width=0.96\textwidth]{../../higgsmassmeasurement/AN-19-248/Figures/Categorisation/Yield_ggF_2017.pdf}
    \captionof{figure}
        [Normalization fit for ggH 2017]
        {Normalization fit for ggH 2017, for different decay channels, as a function of mass, for the 9 bins of $D'^{\text{VXBS}}_{m_{4\ell}}$.}
    \label{signal_normalization_2017}
\end{multiFigure}
\begin{multiFigure}
    \centering
		\includegraphics[width=0.96\textwidth]{../../higgsmassmeasurement/AN-19-248/Figures/Categorisation/Yield_ggF_2018.pdf}
    \captionof{figure}
        [Normalization fit for ggH 2018]
        {Normalization fit for ggH 2018, for different decay channels, as a function of mass, for the 9 bins of $D'^{\text{VXBS}}_{m_{4\ell}}$.}
    \label{signal_normalization_2018}
\end{multiFigure}

\subsubsection{Result using categorization in 1D model}
% TODO: Include the below?
%The mass error uncertainty evaluated in \ref{sec:EBE} is combined with the four-lepton mass to built
%a two-dimentional likelihood function, $\mathcal{L}$($m_{4\ell}$, \massUnc$|m_{H}$), where again 
%$m_{H}$ is fixed to the value of 125 GeV. \\
%\tablename~\ref{table:2D_model_result_year} shows inclusive results compared with 1D result
%for each year.
The expected $\mH$ measurement uncertainty, in case of no-bkg and no-syst,
implementing the categorization,
is reported in \cref{table:2D_model_result} split for different final state and in \cref{table:2D_model_result_year} for different years.
\begin{table}[ht]	
\begin{center}
    \topcaption
        [Expected Higgs boson mass uncertainty measured implementing the categorization,
        for different final states]
        {Expected Higgs boson mass uncertainty measured implementing the categorization,
        for different final state.
        All mass values are given in \MeV.  
        Statistical only results are considered at this stage of the analysis.
        }
    \begin{tabular}{ccccccc}
    \hline			
    Expected uncertainty	&	\fourmu	&	\foure	&	\twoetwomu	&\twomutwoe	& inclusive & Rel. Improvement \\
    \hline			
        N-1$D'_\text{VXBS}$ (No bkg)	&	124	&	319	&	203	&	236	&	92	&	-6\%	\\
        1$D'_\text{VXBS}$ (No bkg)	&	129	&	357	&	223	&	255	&	99	&	-8\%	\\
    %	N-1$D'_\text{VXBS}$ (No bkg)	&	128	&	371	&	227	&	253	&	98	&	-7\%	\\
    %	1$D'_\text{VXBS}$ (No bkg)	&	134	&	424	&	250	&	279	&	105	&	-	\\
    \hline
    %relative improvement	&	-	&	-	&	-	&	-	&	-	\\
    %\hline
    \end{tabular}
\label{table:2D_model_result}
\end{center}
\end{table}
\begin{table}[ht]	
\begin{center}
    \caption
        [Expected Higgs boson mass uncertainty measured implementing the categorization,
        for different years]
        {Expected Higgs boson mass uncertainty measured implementing the categorization,
        for different years.
        All mass values are given in \MeV.  
        Statistical only results are considered at this stage of the analysis.
        }
    \begin{tabular}{ccccc} % TODO: Add total uncertainty in final column on right-hand side?
        \hline			
    Expected uncertainty	&	2016 pre-VFP	&	2016 post-VFP	&	2017	&	2018	\\
        \hline			
        N-1$D'_\text{VXBS}$ (no-bkg)	&	247	&	255	&	170	&	140	\\
        1$D'_\text{VXBS}$ (No bkg)	&	266	&	276	&	182	&	148	\\
    %	N-1$D'_\text{VXBS}$ (no-bkg)	&	192	&	180	&	147	\\
    %	1$D'_\text{VXBS}$ (No bkg)	&	206	&	194	&	157	\\
    \hline
    \end{tabular}
    \label{table:2D_model_result_year}
\end{center}
\end{table}
