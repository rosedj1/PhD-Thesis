% \subsection{\texorpdfstring{\Zone}{Z1} Mass Constraint}  % TODO: Makes Z1 in subsection title NOT bold.
\subsection{Refitting muon and electron \pt with a \texorpdfstring{\Zone}{Z1} Mass Constraint}  % TODO: Makes Z1 in subsection title NOT bold.
% % \subsection{\texorpdfstring{\boldmath{\Zone}}{Z1} Mass Constraint} % TODO: Although \boldmath makes \subsection bold, it also makes the TOC entry bold.
\label{sec:Z1constraint}

%\subsubsection{Z1-mass line shape}
% \subsubsection{Methodology}
% TODO:REWORD, USE CONSISTENT SYMBOLS.
In order to improve the four lepton invariant mass resolution, a kinematic fit is also performed
using a mass constraint on the intermediate on-shell Z resonance, using an approach similar to the one
described previously.
% TODO:CITE
% \cite{RefToZrefit}.
The basic idea is to re-evaluate \pT of two leptons forming 
the Z bosons of the Higgs candidate, with a constraint on the reconstructed Z mass to follow 
the Z boson true lineshape. For a 125 \GeV Higgs, the selected $Z_{1}$ is mostly on-shell, 
while $m_{Z_{2}}$ distribution is broad and the spread is much bigger than detector resolution. 
When considering mass measurment of 125 \GeV Higgs, expected gain in resolution comes from refitting $Z_{1}$.
The likelihood to be maximized can be written as:
\[
\mathcal{L}(p_{T}^{1} , p_{T}^{2}|p_{T}^{reco1}, \sigma p_{T}^{1},p_{T}^{reco2}, \sigma p_{T}^{2}) = 
Gauss(p_{T}^{reco1}|p_{T}^{1}, \sigma p_{T}^{1}) \cdot Gauss(p_{T}^{reco2}|p_{T}^{2}, \sigma p_{T}^{2}) 
\cdot \mathcal{L}(m_{12}|m_{Z},m_{H})
\]
where $p_{T}^{reco1,2}$ are the reconstructed transverse momentum of the two leptons forming the $Z_{1}$,
$\sigma_{p_{T}^{1,2}}$ are the per lepton resolution (uncertainty on \pT measurement, corrected using 
method described in \ref{sec:ebe}), $p_{T}^{1,2}$ are the observables under optimisation,
$m_{12}$ is the invariant mass calculated from $p_{T}^{1}$ and $p_{T}^{2}$. $\mathcal{L}(m_{12}|m_{Z},m_{H})$
is the likelihood, given the true lineshape of $m_{Z_{1}}$. \\
%For a 125 \GeV Higgs, the selected $Z_{1}$ is not always on-shell, so the Breit Wigner shape can not describe
%perfectly its lineshape at generator level. In principle, one can choose any of the $Z_{1}$ 
%lineshapes to be used in the refitting procedure. As long as all Monte Carlor samples and data
%events go through the same procedure (including the same true $Z_{1}$ lineshape), 
%no bias would be introduced.  We choose true gen level $Z_{1}$ lineshape from the SM Higgs boson sample to
%optimize the sensitivity.
For each event, the likelihood is maximized and \pT information of the refitted leptons are updated. \\
Fig.~\ref{fig:Z1_lineshape} shows the on-shell Z resonance line shape, at generator level, 
used to perform the kinematical fit, as taken from ggH 125 \GeV sample. 
Events from all three years have been merged, asking for a 
$p_T > 20(10)$\GeV for (sub)leading leptons, and $|\eta|<$2.4(2.5) for $\mu$(e), focusing on
the reconstructed mass range of [105-140] \GeV in $m_{4l}^{RECO}$. The fit function used is a convolution
of a CB plus three different guassians.
\begin{multiFigure}
    \centering
        \addFigure{0.45}{../../higgsmassmeasurement/AN-19-248/Figures/Z1_lineshape/4mu.pdf}
        \addFigure{0.45}{../../higgsmassmeasurement/AN-19-248/Figures/Z1_lineshape/4e.pdf}
        \addFigure{0.45}{../../higgsmassmeasurement/AN-19-248/Figures/Z1_lineshape/2e2mu.pdf}
        \addFigure{0.45}{../../higgsmassmeasurement/AN-19-248/Figures/Z1_lineshape/2mu2e.pdf}
	\captionof{figure}
        [On-shell Z resonance line shape at generator level, as taken from ggH sample @ 125\GeV]
        {On-shell Z resonance line shape at generator level, as taken from ggH sample @ 125\GeV, % TODO:REWORD
        merging all three years for the 4 different final states.
        \;A) 4$\mu$.
        \;B) 4e.
        \;C) 2e2$\mu$.
        \;D) 2$\mu$2e.}
    \label{fig:Z1_lineshape}
\end{multiFigure}
%After this kinimatic refitting, the mass of the Z candidate, the $m_{4\ell}$, and the $\sigma_{m_{4\ell}}$
%are recalculated. The comparison of the reconstructed and refitted mass can be seen in Fig.~\ref{Z1constraint}.

% \subsection{Mass distributions}
The new invariant mass distributions, with also the constraint of the on-shell Z1 boson, are shown in 
Fig.~\ref{fig:1D_VXBS_Z1_mass_2018_ggH}.
\begin{multiFigure}
    \centering
        \addFigure{0.45}{figures/higgsmassmeas/ggH_MassDistribution/1D_VXBS_Z1_mass_2018_ggH_4mu.pdf}
        \addFigure{0.45}{figures/higgsmassmeas/ggH_MassDistribution/1D_VXBS_Z1_mass_2018_ggH_4e.pdf}
        \addFigure{0.45}{figures/higgsmassmeas/ggH_MassDistribution/1D_VXBS_Z1_mass_2018_ggH_2e2mu.pdf}
        \addFigure{0.45}{figures/higgsmassmeas/ggH_MassDistribution/1D_VXBS_Z1_mass_2018_ggH_2mu2e.pdf}
    \captionof{figure}
        [Four-lepton invariant mass distribution, with VXBS and Z1 constraint]
        {Four-lepton invariant mass distribution, with VXBS and Z1 constraint, in the signal region ([105-140]\GeV) using ggH events, with the DSCB fit for 2018: % TODO:REWORD
        \;A) 4$\mu$.
        \;B) 4e.
        \;C) 2e2$\mu$.
        \;D) 2$\mu$2e.} 
    \label{fig:1D_VXBS_Z1_mass_2018_ggH}
\end{multiFigure}
Comparing these new distributions (Fig.~\ref{fig:1D_VXBS_Z1_mass_2018_ggH}) with the previous ones (Fig.~\ref{fig:1D_VXBS_mass_2018_ggH}), it can be seen that the \Zone constraint has a bigger improvement in final states with on-shell Z boson decaying in to 2e (~30$\%$ improvement on $\sigma$ for 2e2$\mu$ and 16$\%$ for 4e) while the other two final states (4$\mu$ and 2$\mu$2e) are less affected (improvement in $\sigma$ of the other of 5$\%$).

\subsubsection{Expected mH measurement uncertainties (MC) and relative improvements}
The new four-lepton mass ($m'_{4\ell}$), shown in Fig.~\ref{fig:1D_VXBS_Z1_mass_2018_ggH},
is used to rebuild the 1D likelihood function, $\mathcal{L}$($m'_{4\ell}|m_{H}$). Signal normalisation and signal parameterization are extracted following the procedure described in \ref{sec:signal_model}.
The expected \mH measurement uncertainty, is reported in Table~\ref{table:1D_model_result_Z1} (or in Table~\ref{table:1D_model_result_Z1_year}
splitted in years).
%===
\begin{table}[ht]	
\begin{center}
    \topcaption
        [Expected Higgs boson mass uncertainty measured with 1D model, with and without
        Z1 refit for different final states]
        {Expected Higgs boson mass uncertainty measured with 1D model, with and without
        Z1 refit for different final states. All mass values are given in \MeV.
        Statistical only results are considered at this stage of the analysis.
        }
    \begin{tabular}{ccccccc}
        \hline			
    Expected uncertainty	&	4$\mu$	&	4e	&	2e2$\mu$	&2$\mu$2e	& inclusive & Rel. Improvement \\
        \hline			
        1$D'_{VXBS}$ (No bkg)	&	129	&	357	&	223	&	255	&	99	&	-8\%	\\
        1$D_{VXBS}$  (No bkg)	&	137	&	394	&	275	&	266	&	108 &	-4\%	\\
    %	1$D'_{VXBS}$ (No bkg)	&	134	&	424	&	250	&	279	&	105	&	-9\%	\\
    %	1$D_{VXBS}$  (No bkg)	&	144	&	466	&	313	&	291	&	116	&	-\\	
        %\hline
    %relative improvement	&	-	&	-	&	-	&	-	&	-	\\
        \hline
\end{tabular}
\label{table:1D_model_result_Z1}
\end{center}
\end{table}
%===
\begin{table}[ht]	
\begin{center}
    \topcaption
        [Expected Higgs boson mass uncertainty measured with 1D model, with and without
        Z1 refit for different years]
        {Expected Higgs boson mass uncertainty measured with 1D model, with and without
        Z1 refit for different years. All mass values are given in \MeV.
        Statistical only results are considered at this stage of the analysis.
        }
    \begin{tabular}{ccccc}
        \hline			
    Expected uncertainty	&	2016 pre-VFP	&	2016 post-VFP	&	2017	&	2018	\\
        \hline			
        1$D'_{VXBS}$ (No bkg)	&	266	&	276	&	182	&	148	\\
        1$D_{VXBS}$	(No bkg)	&	291	&	301	&	198	&	162	\\
    %	1$D'_{VXBS}$ (No bkg)	&	206	&	194	&	157	\\
    %	1$D_{VXBS}$	(No bkg)	&	227	&	212	&	173	\\
        \hline
    \end{tabular}
    \label{table:1D_model_result_Z1_year}
\end{center}
\end{table}
