\section{Analysis Overview}
\label{sec:analysis_overview}
The first step to performing a precision measurement of the Higgs boson mass is to ``observe'' many Higgs bosons.
Since the Higgs boson has a \emph{very} short mean lifetime of only $1 \tentotheminus{23}\snd$~\cite{pdg},


- Want to measure the Higgs boson mass (mH), so need Higgs bosons.
- Sift through CMS data for \hzzfourl events (the \emph{signal} process) because S B ratio is huge: 2.
    - However \hzzfourl is rare:
Although the LHC , the Higgs boson is produced in only 1 out of every billion pp collisions.
Even if \PH is produced, it will decay into two \PZ only a small percentage of the time (2.3\%).
This percentage is typically expressed as a fraction, called the \emph{branching fraction} or \emph{branching ratio} (\br).
there is a small probability of only  that it will decay into two \PZ bosons (2.3\%).
Those \PZ bosons then have only a small probability 
so the boson itself will never live long enough to interact directly with the CMS Detector.
Therefore, the Higgs boson can only be detected by the daughter particles into which it decays.
- By collecting events with the \fourl final state, we are likely to find signal events.
    - It's not just the signal process which produces \fourl: background also makes \fourl (Section FIXME).
- Before analyzing the data, however, it is important to make predictions using simulated samples (Section FIXME).
- In order to sort signal from background, use simulated samples 

- Form objects from data.
- Use objects and conservation of momentum to rebuild parent particles.
    - The \PZ boson has a precisely measured mass of TODO a neutral particle, so the two leptons into which it decays should combine to Group two leptons together, 
    - Form two different pairs of opposite-sign, same-flavor (OSSF) leptons
    - If it appears that the to select specific hzz4l events (\emph{event selection}).
- 


