\section{Signal Modeling}
\label{sec:signal_model}

\subsection{Signal Normalization}
\label{sec:SignalNormalization}
The normalization of the Higgs boson signal is obtained from simulation, 
by looking at the expected signal yields within the range 105--140\GeV, 
using five simulated mass points: $\mH = 120, 124, 125, 126, 130\GeV$.
A second order polynomial function is used to extract the dependence of the normalization from \mH.
Fits are performed separately for each production mode, for each decay channel, and for each year. 
Examples of the fits are shown in \cref{fig:signal_normalization}.
\begin{multiFigure}
    \centering
        \addFigure{0.48}{../../higgsmassmeasurement/AN-19-248/Figures/SignalModelling/Signal_Normalisation/NoZ1/2016pre_ggH_yield_workinprogress.pdf}
        \addFigure{0.48}{../../higgsmassmeasurement/AN-19-248/Figures/SignalModelling/Signal_Normalisation/NoZ1/2016post_ggH_yield_workinprogress.pdf}
        \addFigure{0.48}{../../higgsmassmeasurement/AN-19-248/Figures/SignalModelling/Signal_Normalisation/NoZ1/2017_ggH_yield_workinprogress.pdf}
        \addFigure{0.48}{../../higgsmassmeasurement/AN-19-248/Figures/SignalModelling/Signal_Normalisation/NoZ1/2018_ggH_yield_workinprogress.pdf}
    \captionof{figure}
        [Normalization fit for the ggH signal in different final states]
        {Normalization fit for the ggH signal in different final states (expected yield \vs $\mH / 125\GeV$) for all years during Run 2.
        \;A) 2016 pre-VFP.
        \;B) 2016 post-VFP.
        \;C) 2017.
        \;D) 2018.}
    \label{fig:signal_normalization}
\end{multiFigure}

\subsection{Parameterizing the Signal Line Shape}
\label{sec:SignalParametrization}
The signal lineshape is obtained from the fit of the \mfourl distribution in the range 105--140\GeV, using a double-sided Crystal Ball (DSCB) function, which has 6 parameters.
Fit parameters $\left( \theta^i \right)$ are derived as a function of \mH, using a first-order polynomial:
\[
\theta^{i}_\text{DSCB} = a^i + b^{i}~(\mH - 125)% + c~(m_{H} - 125)^{2}
\]
The process begins by fitting only the 125\GeV sample, during which the $a^i$ term for each parameter is extracted ($b^i$ is not yet taken into account).
Then, a second fit is performed, during which $a^i$ is fixed to the previously value found, while $b^i$ is obtained while simultaneously fitting all five \mH mass points.

The fit is performed separately, for each production mode, for each decay channel, for each year. 
To take into account the non-resonant contribution of $\PV\PH$ and $\ttbarh$ production modes,
the DSCB is convoluted with a Landau function that describes the possibility for a lepton from the Higgs boson decay to be lost or to not be selected.

\subsection{Building the 1D PDF}
The measurement of \mH is extracted by maximizing the one-dimensional likelihood function $\lhood \left( \mH \middlepipe \mfourl \right)$ across the \mfourl spectrum.
The model and the normalization used for the signal are described in Sec.~\ref{sec:SignalNormalization}.
The DSCB fits of the four-lepton invariant mass distributions are shown in \cref{fig:1D_mass_2018_ggH} for the signal region (105--140\GeV) using simulated ggH events, split into the four different final states.
\begin{multiFigure}
    \centering
        \addFigure{0.48}{../../higgsmassmeasurement/AN-19-248/Figures/ggH_MassDistribution/1D_mass_2018_ggH_4mu_workinprogress.pdf}
        \addFigure{0.48}{../../higgsmassmeasurement/AN-19-248/Figures/ggH_MassDistribution/1D_mass_2018_ggH_4e_workinprogress.pdf}
        \addFigure{0.48}{../../higgsmassmeasurement/AN-19-248/Figures/ggH_MassDistribution/1D_mass_2018_ggH_2e2mu_workinprogress.pdf}
        \addFigure{0.48}{../../higgsmassmeasurement/AN-19-248/Figures/ggH_MassDistribution/1D_mass_2018_ggH_2mu2e_workinprogress.pdf}
    \captionof{figure}
        [Distributions of \mfourl in the signal region (105--140\GeV) using simulated ggH events]
        {Distributions of \mfourl in the signal region (105--140\GeV) using simulated ggH events, fitted with a DSCB function for the year 2018.
        \;A) The \fourmu final state.
        \;B) The \foure final state.
        \;C) The \twoetwomu final state.
        \;D) The \twomutwoe final state.}
    \label{fig:1D_mass_2018_ggH}
\end{multiFigure}

\subsubsection{Expected Uncertainties on \mH before Improvements}
Assuming perfect background rejection and neglecting any systematic uncertainties, the expected uncertainty on the measurement of \mH is shown in \cref{table:1D_model_result_fs,table:1D_model_result_year}, split by final state and by year, respectively.
% TODO: Consider using tabularx
%=== By final state.
\begin{table}[!ht]
    % \small
    \centering
    \topcaption
        [Expected Higgs boson mass uncertainty measured with the 1D model]
        {Expected Higgs boson mass uncertainty measured with the 1D model for different final states.
        All mass uncertainties are reported in $\MeVns$ and are statistical-only.}
	\begin{tabular}{cccccc}
            \hline      
        Expected uncertainty $(\MeVns)$	&	\fourmu	&	\foure	&	\twoetwomu	& \twomutwoe	& Inclusive	\\
            \hline
        1D	(No bkg) &	147	&	394	&	276	&	273	&	112	\\
            \hline
        \end{tabular}
    \label{table:1D_model_result_fs}
\end{table}
%=== By year.
\begin{table}[!ht]
    % \small
    \centering
    \topcaption
        [Expected Higgs boson mass uncertainty measured with the 1D model]
        {Expected Higgs boson mass uncertainty measured with the 1D model for different years during Run 2.
        All mass uncertainties are reported in $\MeVns$ and are statistical-only.
        }
	\begin{tabular}{|ccccc|}
            \hline      
        Expected uncertainty $(\MeVns)$	&	2016 pre-VFP	&	2016 post-VFP	&	2017	&	2018	\\
            \hline
        1D (No bkg)	&	304	&	314	&	207	&	170	\\
            \hline
        \end{tabular}
    \label{table:1D_model_result_year}
\end{table}