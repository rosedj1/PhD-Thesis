%=== SigFigs on integrated luminosity values.
% The current PC recommendation is to show three digits for any integrated luminosity value, i.e., 138/fb for 2016--2018 and 36.3/fb for 2016-only, even when both numbers are quoted in the same document. 
%=== L_int Uncertainty ===% 
% From: https://twiki.cern.ch/twiki/bin/view/CMS/TWikiLUM#Quick_summary_table
%                                       2015        2016    	           2017 	        2018 	        2015-2018 	            2016-2018
% Total delivered luminosity [1/fb] 	4.31 	    41.5     	           49.81 	        67.86 	        163.56          	    159.25
% Recommended luminosity [1/fb] 	    2.27† 	    36.3 (36.31)¶ 	       41.48 (41.53) 	59.83 (59.74) 	139.92 (139.87) 	    137.65 (137.60)
% Preliminary luminosity [1/fb] 	    2.26†§ 	    35.9 (35.92)§          41.48 (41.53) 	59.83 (59.74)   139.50 (139.45)§ 	    137.24 (137.19)§
% Change of recommended/preliminary 	0.7% 	    1.1%    	           ― 	             ― 	            0.3 (0.3) % 	        0.3 (0.3) %
% Recommended uncertainty [%] 	        1.6 	    1.2     	           2.3 	            2.5 	        1.6 	                1.6
% Reference(s) 	                        LUM-17-003 	LUM-17-003 	           LUM-17-004 	    LUM-18-002 	    LUM-17-003, LUM-18-002 	LUM-17-003, LUM-18-002 
% Always include the reference to the 2015/2016 pp 13 TeV luminosity paper (LUM-17-003), which describes the methodology used by all luminosity measurements also for other datasets (i.e. also for pp 5 TeV, pPb, PbPb analyses).
% Additionally, include references to the PASes that describe the luminosity measurement of the used data sets, e.g. LUM-17-004 (2017) and LUM-18-002 (2018) for full Run 2 pp 13 TeV analyses. 
\section{Systematic Uncertainties}
\label{sec:syst_uncert}
The systematic uncertainties involved in this analysis are detailed in TODO:CITE HIG-19-001.
For reference, the systematic uncertainties are summarized below:
\begin{itemize}
	\item $\lumiint$ (1.6\%)~\cite{LUM-17-003, LUM-17-004, LUM-18-002}.
	\item Lepton momentum scale and resolution (conservative values from extensive analysis are given in \tablename~\ref{table:ScaleRes_syst}.)
	\item Lepton identification and reconstruction efficiencies (2.5--9\%), depending on the final state % TODO: confirm number.
	\item Lepton energy scale
	% \item BOTH OF THE ABOVE AFFECT SIGNAL AND BKG.
	\item Estimation of the reducible background (31\%) TODO: cite.
\end{itemize}
\begin{table}[ht]
	\begin{center}
		\topcaption
		% []
		{Impact of the lepton scale and resolution.}
                \begin{tabular}{ccccc}
		\hline
		Syst	&	4$\mu$	&	4e	&	2e2$\mu$	&	2$\mu$2e	\\
		\hline
		Muon scale	&	0.01\%	&	---	&	0.01\%	&	0.01\%	\\
		Electron scale	&	---	&	0.06\%	&	0.06\%	&	0.06\%	\\
		\hline
		Muon resoltuon	&	3\%	&	---	&	3\%	&	3\%	\\
		Electron resolution	&	---	&	10\%	&       10\%&       10\%	\\
		\hline
		\end{tabular}
		\label{table:ScaleRes_syst}
	\end{center}
\end{table}

The theoretical systematic uncertainties include:
\begin{itemize}
	\item Renormalization and factorization scale uncertainties  TODO: Get number and cite.
	\item Choice of the set of parton distribution functions TODO: Get number and cite.
	% \item BOTH OF THE ABOVE AFFECT SIGNAL AND BKG
	\item Additional uncertainty of 10\% on the $K$ factor used for \ggzzfourl prediction is applied
	\item Branching fraction uncertainties for signal and background processes TODO: Get number and cite.
\end{itemize}
and are summarized in \tablename~\ref{table:theo_syst}.
A systematic uncertainty of 2\% on the branching fraction of \hzzfourl only affects the signal yield is also considered. 
\begin{table}[ht]	
	\begin{center}
		\topcaption
		% TODO: []
		{Summary of the systematic uncertainties.}
		\begin{tabular}{cccc}
		\hline
		\multicolumn{4}{c}{Theory uncertainties} \\
		\hline
		Name & 2016 & 2017 & 2018 \\
		\hline
		\hline
		QCD scale ggH	& \multicolumn{3}{c}{$\pm$3.9\%} \\
		QCD scale VBF	& \multicolumn{3}{c}{+0.4/-0.3\%} \\
		QCD scale WH	& \multicolumn{3}{c}{+0.5/-0.7\%} \\
		QCD scale ZH	& \multicolumn{3}{c}{+3.8/-3.1\%} \\
		QCD scale ttH	& \multicolumn{3}{c}{+5.8/-9.2\%} \\
		\hline
		\hline
		QCD scale qqZZ	& \multicolumn{3}{c}{+3.2/-4.2\%} \\
		QCD scale ggZZ	& \multicolumn{3}{c}{$\pm$3.9\%} \\
		\hline
		\hline
		PDF set gg	& \multicolumn{3}{c}{$\pm$3.2\%} \\
		PDF set qq	& \multicolumn{3}{c}{$\pm$2.1\%} \\
		\hline
		\end{tabular}
		\label{table:theo_syst}
	\end{center}
\end{table}
