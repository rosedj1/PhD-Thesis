%=== SigFigs on integrated luminosity values.
% The current PC recommendation is to show three digits for any integrated luminosity value, i.e., 138/fb for 2016--2018 and 36.3/fb for 2016-only, even when both numbers are quoted in the same document. 
%=== L_int Uncertainty ===% 
% From: https://twiki.cern.ch/twiki/bin/view/CMS/TWikiLUM#Quick_summary_table
%                                       2015        2016    	           2017 	        2018 	        2015-2018 	            2016-2018
% Total delivered luminosity [1/fb] 	4.31 	    41.5     	           49.81 	        67.86 	        163.56          	    159.25
% Recommended luminosity [1/fb] 	    2.27† 	    36.3 (36.31)¶ 	       41.48 (41.53) 	59.83 (59.74) 	139.92 (139.87) 	    137.65 (137.60)
% Preliminary luminosity [1/fb] 	    2.26†§ 	    35.9 (35.92)§          41.48 (41.53) 	59.83 (59.74)   139.50 (139.45)§ 	    137.24 (137.19)§
% Change of recommended/preliminary 	0.7% 	    1.1%    	           ― 	             ― 	            0.3 (0.3) % 	        0.3 (0.3) %
% Recommended uncertainty [%] 	        1.6 	    1.2     	           2.3 	            2.5 	        1.6 	                1.6
% Reference(s) 	                        LUM_17_003 	LUM_17_003 	           LUM_17_004 	    LUM_18_002 	    LUM_17_003, LUM_18_002 	LUM_17_003, LUM_18_002 
% Always include the reference to the 2015/2016 pp 13 TeV luminosity paper (LUM_17_003), which describes the methodology used by all luminosity measurements also for other datasets (i.e. also for pp 5 TeV, pPb, PbPb analyses).
% Additionally, include references to the PASes that describe the luminosity measurement of the used data sets, e.g. LUM_17_004 (2017) and LUM_18_002 (2018) for full Run 2 pp 13 TeV analyses. 

%=== 2018 Prompt Reco
% https://twiki.cern.ch/twiki/bin/view/CMS/PdmV2018Analysis
% 			Absolute Run Number 	Collision Runs Only 	Energy 	Dataset name 						Json Lumi (/fb) (Without normtag)	Json Lumi (/fb) (With normtag)
% Run2018A 	315252 	316995 			315252 	316995 			13 		/PromptReco/Collisions2018A/DQM 	13.48 /fb 							14.00 /fb 	
% Run2018B 	316998 	319312 			317080 	319310 			13 		/PromptReco/Collisions2018B/DQM 	6.785 /fb 							7.10 /fb 	
% Run2018C 	319313 	320393 			319337 	320065 			13 		/PromptReco/Collisions2018C/DQM 	6.612 /fb 							6.94 /fb 	
% Run2018D 	320394 	325273 			320673 	325175 			13 		/PromptReco/Collisions2018D/DQM 	31.95 /fb 							31.93 /fb 

\section{Systematic Uncertainties}
\label{sec:syst_uncert}
The sources of systematic uncertainties and the impact they have on the estimation of \mfourl are detailed in Ref.~\cite{HIG_21_019}, which include
\begin{itemize}
	\item integrated luminosity, $\lumiint$ (1.6\%)~\cite{LUM_17_003, LUM_17_004, LUM_18_002}
	\item lepton momentum scale (0.01--0.06\%) and momentum resolution (3--10\%), depending on the final state (\tablename~\ref{table:ScaleRes_syst})
	\item lepton identification and reconstruction efficiencies (2.5--9\%), depending on the final state
	\item estimation of the reducible background ($\sim$31\%), depending on the final state.
\end{itemize}
\begin{table}[ht]
	\begin{center}
		\topcaption
		% [] % TODO
		{Impact of the lepton momentum scale and resolution.}
                \begin{tabular}{ccccc}
		\hline
		Syst	&	4$\mu$	&	4e	&	2e2$\mu$	&	2$\mu$2e	\\
		\hline
		Muon scale	&	0.01\%	&	---	&	0.01\%	&	0.01\%	\\
		Electron scale	&	---	&	0.06\%	&	0.06\%	&	0.06\%	\\
		\hline
		Muon resolution	&	3\%	&	---	&	3\%	&	3\%	\\
		Electron resolution	&	---	&	10\%	&       10\%&       10\%	\\
		\hline
		\end{tabular}
		\label{table:ScaleRes_syst}
	\end{center}
\end{table}
The sources of theoretical uncertainties come from Ref.~\cite{HIG_19_001},
are summarized in \tablename~\ref{table:theo_syst},
and have impacts on \mfourl which include
\begin{itemize}
	\item renormalization and factorization scale uncertainties come from varying these scales between 0.5--2.0 times their nominal values
	\item choice of the set of parton distribution functions~\cite{Alekhin:2011sk, Botje:2011sn}
	\item additional uncertainty of 10\% on the $K$ factor used for \ggzzfourl prediction
	\item branching fraction uncertainties for \hzzfourl signal (2\%).
\end{itemize}
\begin{table}[ht]	
	\begin{center}
		\topcaption
			[Summary of the systematic uncertainties used in the measurement of \mH.] % TODO: REWORD?
			{Summary of the systematic uncertainties used in the measurement of \mH.}
		\begin{tabular}{cccc}
		\hline
		\multicolumn{4}{c}{Theory uncertainties} \\
		\hline
		Name & 2016 & 2017 & 2018 \\
		\hline
		\hline
		QCD scale ggH	& \multicolumn{3}{c}{$\pm$3.9\%} \\
		QCD scale VBF	& \multicolumn{3}{c}{+0.4/-0.3\%} \\
		QCD scale WH	& \multicolumn{3}{c}{+0.5/-0.7\%} \\
		QCD scale ZH	& \multicolumn{3}{c}{+3.8/-3.1\%} \\
		QCD scale ttH	& \multicolumn{3}{c}{+5.8/-9.2\%} \\
		\hline
		\hline
		QCD scale qqZZ	& \multicolumn{3}{c}{+3.2/-4.2\%} \\
		QCD scale ggZZ	& \multicolumn{3}{c}{$\pm$3.9\%} \\
		\hline
		\hline
		PDF set gg	& \multicolumn{3}{c}{$\pm$3.2\%} \\
		PDF set qq	& \multicolumn{3}{c}{$\pm$2.1\%} \\
		\hline
		\end{tabular}
		\label{table:theo_syst}
	\end{center}
\end{table}
