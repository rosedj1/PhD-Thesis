\section{Techniques to Improve Precision on \mH Measurement}
\label{sec:techniques}

% %=== Hualin's thesis ===%
% % \subsubsection{Computation of Per-Event Mass Uncertainties}
% % \subsubsection{Correction of Lepton Momentum Uncertainties}
% % \subsubsection{Model and Procedure to Derive Corrections}
% % \subsubsection{Validation of Corrections}
% %=== AN-19-248 ===%
% % \subsubsection{motivation}
% % \subsubsection{model and procedure to derive Corrections}
% % \subsubsection{validation of corrections (MC, data)}

% \subsection{Matrix Element-Based Kinematic Discriminant}

% \begin{table}[ht]
%     \begin{center}
%     \begin{tabular}{|c|cccc|c|}
%     \hline			
%     Expected uncertainty	&	4$\mu$	&	4e	&	2e2$\mu$	&2$\mu$2e	& Inclusive	\\
%     \hline			
%         1D	(No bkg) &	153	&	466	&	315	&	300	&	121	\\
%     \hline
%     \end{tabular}
%     \end{center}
%     \caption{
%         %=== Below is the description from AN-19-248. ==%
%         % Expected Higgs boson mass uncertainty measured with 1D model 
%         % for different final states. All mass values are given in \MeV.
%         % Statistical-only results are considered at this stage of the analysis.
%         }
%     \label{table:model_result_fs_1D}
%     \end{table}

Several analysis techniques---some old and some new with respect to the 2016 measurement of \mH by CMS TODO:Ref HIG-16-041---help to improve the precision on \mH.
Each technique is individually described in the sections that follow.

% TODO: make mfourl bold here but not in TOC.
% TODO: check out sec-075-newEBE.tex
% \subsection{Per-Event Relative Mass Error Categorization}
\subsection{Improving Event-by-Event \mfourl Uncertainty}
\label{sec:ebe}
% TODO:REWORD
For each event, the uncertainty on the \mfourl value (\mfourlerr) is used directly in the \Zone constraint (Sec.~\ref{sec:Z1constraint}).
By improving the precision on the \mfourlerr 
Per-event four-lepton uncertainties are used for (a) performing the \Zone constraint and (b) in the measurement of \mH so that better-measured events are given higher weights in the likelihood fit.
Individual lepton uncertainty on momentum measurement can be predicted on a per-lepton 
basis. In the case of muons, the full error matrix is obtained using muon track fit; for the electrons,
instead, the momentum error is estimated from the combination of the ECAL and tracker measurement, 
neglecting the uncertainty on the track direction from the GSF fit. \\
The uncertainty on the kinematics at the per-lepton level is then propagated to the four-lepton case 
to predict the mass error on an event-by-event basis, using the following approach.\\
Each $\delta m_{i}$, corresponding to individual lepton momentum variation, is calculated separately 
and then the measured resolution on the invariant mass of the four leptons is taken as the quadrature sum 
of the four individual $\delta m_{i}$:
\[
m_{0} = F(p_{T1}, \phi_{1},\eta_{1}; p_{T2}, \phi_{2},\eta_{2}; p_{T3}, \phi_{3}, \eta_{3}; p_{T4}, \phi_{4},\eta_{4})
\]
\[\delta m_{i} = F(...; p_{Ti} + \delta p_{Ti}, \phi_{i}, \eta_{i}; ...) - m_{0} \quad
\]
\[
\delta m = \sqrt{\delta m_{1}^2 + \delta m_{2}^2 + \delta m_{3}^2 + \delta m_{4}^2}
\]

%=== Can't get the below to be centered...
% \begin{align*}
%         m_0 = F(
%         p_{T1}, \phi_1, \eta_1;
%         p_{T2}, \phi_2, \eta_2;
%         p_{T3}, \phi_3, \eta_3;
%         p_{T4}, \phi_4, \eta_4
%         )
%         \\
%         \delta m_i = F(...; p_{Ti} + \delta p_{Ti}, \phi_i, \eta_i; ...) - m_0
%         \\
%         \delta m = \sqrt{
%         (\delta m_1)^2 + (\delta m_2)^2 + (\delta m_3)^2 + (\delta m_4)^2
%         }
% \end{align*}
% TODO: Check with Filippo if these are MC or data.
The full error matrices ($\delta p_{T}/p_{T}$, $\eta$) for muons and electrons, separately, are shown in Fig.~\ref{fig:2D_Mpas_vs_eta} for all years.
\begin{multiFigure}
    \centering

    \addFigure{0.32}{../../higgsmassmeasurement/AN-19-248/Figures/EBE/2016_vs_eta_muon_workinprogress.pdf}
    \addFigure{0.32}{../../higgsmassmeasurement/AN-19-248/Figures/EBE/2016_vs_eta_ele_ECAL_workinprogress.pdf}
    \addFigure{0.32}{../../higgsmassmeasurement/AN-19-248/Figures/EBE/2016_vs_eta_ele_tracker_workinprogress.pdf}

    \addFigure{0.32}{../../higgsmassmeasurement/AN-19-248/Figures/EBE/2017_vs_eta_muon_workinprogress.pdf}
    \addFigure{0.32}{../../higgsmassmeasurement/AN-19-248/Figures/EBE/2017_vs_eta_ele_ECAL_workinprogress.pdf}
    \addFigure{0.32}{../../higgsmassmeasurement/AN-19-248/Figures/EBE/2017_vs_eta_ele_tracker_workinprogress.pdf}

    \addFigure{0.32}{../../higgsmassmeasurement/AN-19-248/Figures/EBE/2018_vs_eta_muon_workinprogress.pdf}
    \addFigure{0.32}{../../higgsmassmeasurement/AN-19-248/Figures/EBE/2018_vs_eta_ele_ECAL_workinprogress.pdf}
    \addFigure{0.32}{../../higgsmassmeasurement/AN-19-248/Figures/EBE/2018_vs_eta_ele_tracker_workinprogress.pdf}
    \captionof{figure}
        [Scatterplots of the relative \pt error \vs $\eta$]
        {Scatterplots of the relative \pt error \vs $\eta$ in data.
        \;Left column: A, D, G) Muons.
        \;Middle column: B, E, H) ECAL-driven electrons.
        \;Right column: C, F, I) Tracker-driven electrons.
        \;Top row: A, B, C) 2016.
        \;Middle row: D, E, F) 2017.
        \;Bottom row: G, H, I) 2018.
        }
    \label{fig:2D_Mpas_vs_eta}
\end{multiFigure}
% \begin{figure}[!htbp]
% 	\begin{center}
% %		\subfloat[][2016]
% %		   {		
%                     \includegraphics[width=0.32\textwidth]{Figures/EBE/2016_vs_eta_muon.pdf}
% 					\includegraphics[width=0.32\textwidth]{Figures/EBE/2016_vs_eta_ele_ECAL.pdf} 
% 					\includegraphics[width=0.32\textwidth]{Figures/EBE/2016_vs_eta_ele_tracker.pdf}
% %			}\\
% %		\subfloat[][2017]
%             % {
% 		    		\includegraphics[width=0.32\textwidth]{Figures/EBE/2017_vs_eta_muon.pdf}
% 					\includegraphics[width=0.32\textwidth]{Figures/EBE/2017_vs_eta_ele_ECAL.pdf} 
% 					\includegraphics[width=0.32\textwidth]{Figures/EBE/2017_vs_eta_ele_tracker.pdf}
% %			}\\
% %		\subfloat[][2018]
%             % {
% 		    		\includegraphics[width=0.32\textwidth]{figures/higgsmassmeas/ebe/2018_vs_eta_muon.pdf}
% 					\includegraphics[width=0.32\textwidth]{figures/higgsmassmeas/ebe/2018_vs_eta_ele_ECAL.pdf} 
% 					\includegraphics[width=0.32\textwidth]{figures/higgsmassmeas/ebe/2018_vs_eta_ele_tracker.pdf}
% %			}\\
% 		\caption{Scatterplot of the relative lepton \PT error vs $\eta$ for muons (left), 
% 				ECAL driven electrons (middle), and tracker driven electrons (right) for 2018 data.}
% 	\label{fig:2D_Mpas_vs_eta}
% 	\end{center}
% \end{figure}

Starting from these distributions, corrections to lepton momentum uncertainty in mutual $|\eta|$ bins are derived for muons (Fig.~\ref{fig:2D_Mpas_vs_pt_muon}), ECAL-driven electrons (Fig.~\ref{fig:2D_Mpas_vs_pt_electron_ECAL}), and tracker electrons (Fig.~\ref{fig:2D_Mpas_vs_pt_electron_tracker}) using bins of $\delta p_{T}$/$p_{T}$ \vs \abseta.
Scatterplots of $\delta p_{T}$/$p_{T}$ \vs \PT are shown in Figs.~\ref{fig:2D_Mpas_vs_pt_electron_ECAL} and~\ref{fig:2D_Mpas_vs_pt_electron_tracker}.
% Muons.
\begin{multiFigure}
    \centering
    \addFigure{0.32}{../../higgsmassmeasurement/AN-19-248/Figures/EBE/2016_vs_pt_muon_1_workinprogress.pdf}
    \addFigure{0.32}{../../higgsmassmeasurement/AN-19-248/Figures/EBE/2016_vs_pt_muon_2_workinprogress.pdf}
    \addFigure{0.32}{../../higgsmassmeasurement/AN-19-248/Figures/EBE/2016_vs_pt_muon_3_workinprogress.pdf}

    \addFigure{0.32}{../../higgsmassmeasurement/AN-19-248/Figures/EBE/2017_vs_pt_muon_1_workinprogress.pdf}
    \addFigure{0.32}{../../higgsmassmeasurement/AN-19-248/Figures/EBE/2017_vs_pt_muon_2_workinprogress.pdf}
    \addFigure{0.32}{../../higgsmassmeasurement/AN-19-248/Figures/EBE/2017_vs_pt_muon_3_workinprogress.pdf}

    \addFigure{0.32}{../../higgsmassmeasurement/AN-19-248/Figures/EBE/2018_vs_pt_muon_1_workinprogress.pdf}
    \addFigure{0.32}{../../higgsmassmeasurement/AN-19-248/Figures/EBE/2018_vs_pt_muon_2_workinprogress.pdf}
    \addFigure{0.32}{../../higgsmassmeasurement/AN-19-248/Figures/EBE/2018_vs_pt_muon_3_workinprogress.pdf}
    \captionof{figure}
        [Scatterplots of the relative \pT error vs \pT for muons in bins of \abseta]
        % TODO: fix words below.
        {Scatterplots of the relative \pT error vs \pT for muons. $0 < \abseta < 0.9$ (left column), $0.9 < \abseta < 1.8$ (middle column), and $1.8 < \abseta < 2.4$ (right column) for 2016 (top row), 2017 (middle row), and 2018 (bottom row) data.}
    \label{fig:2D_Mpas_vs_pt_muon}
\end{multiFigure}
% ECAL electrons.
\begin{multiFigure}
    \centering
    \addFigure{0.32}{../../higgsmassmeasurement/AN-19-248/Figures/EBE/2016_vs_pt_ECAL_1_workinprogress.pdf}
    \addFigure{0.32}{../../higgsmassmeasurement/AN-19-248/Figures/EBE/2016_vs_pt_ECAL_2_workinprogress.pdf}
    \addFigure{0.32}{../../higgsmassmeasurement/AN-19-248/Figures/EBE/2016_vs_pt_ECAL_3_workinprogress.pdf}

    \addFigure{0.32}{../../higgsmassmeasurement/AN-19-248/Figures/EBE/2017_vs_pt_ECAL_1_workinprogress.pdf}
    \addFigure{0.32}{../../higgsmassmeasurement/AN-19-248/Figures/EBE/2017_vs_pt_ECAL_2_workinprogress.pdf}
    \addFigure{0.32}{../../higgsmassmeasurement/AN-19-248/Figures/EBE/2017_vs_pt_ECAL_3_workinprogress.pdf}

    \addFigure{0.32}{../../higgsmassmeasurement/AN-19-248/Figures/EBE/2018_vs_pt_ECAL_1_workinprogress.pdf}
    \addFigure{0.32}{../../higgsmassmeasurement/AN-19-248/Figures/EBE/2018_vs_pt_ECAL_2_workinprogress.pdf}
    \addFigure{0.32}{../../higgsmassmeasurement/AN-19-248/Figures/EBE/2018_vs_pt_ECAL_3_workinprogress.pdf}
    \captionof{figure}
        [Scatterplots of the relative \pT error vs \pT for ECAL-driven electrons in bins of \abseta]
        % TODO: Is it OK to skip over 1.0--2.0?
        {Scatterplots of the relative \pT error vs \pT for ECAL-driven electrons with $0 < \abseta < 0.8$ (left column), $0.8 < \abseta < 1.0$ (middle column), and $2.0 < \abseta < 2.5$ (right column) for 2016 (top row), 2017 (middle row), and 2018 (bottom row) data.}
    \label{fig:2D_Mpas_vs_pt_electron_ECAL}
\end{multiFigure}
% Tracker electrons.
\begin{multiFigure}
    \centering
    \addFigure{0.32}{../../higgsmassmeasurement/AN-19-248/Figures/EBE/2016_vs_pt_tracker_1_workinprogress.pdf}
    \addFigure{0.32}{../../higgsmassmeasurement/AN-19-248/Figures/EBE/2016_vs_pt_tracker_2_workinprogress.pdf}
    \addFigure{0.32}{../../higgsmassmeasurement/AN-19-248/Figures/EBE/2016_vs_pt_tracker_3_workinprogress.pdf}

    \addFigure{0.32}{../../higgsmassmeasurement/AN-19-248/Figures/EBE/2017_vs_pt_tracker_1_workinprogress.pdf}
    \addFigure{0.32}{../../higgsmassmeasurement/AN-19-248/Figures/EBE/2017_vs_pt_tracker_2_workinprogress.pdf}
    \addFigure{0.32}{../../higgsmassmeasurement/AN-19-248/Figures/EBE/2017_vs_pt_tracker_3_workinprogress.pdf}

    \addFigure{0.32}{../../higgsmassmeasurement/AN-19-248/Figures/EBE/2018_vs_pt_tracker_1_workinprogress.pdf}
    \addFigure{0.32}{../../higgsmassmeasurement/AN-19-248/Figures/EBE/2018_vs_pt_tracker_2_workinprogress.pdf}
    \addFigure{0.32}{../../higgsmassmeasurement/AN-19-248/Figures/EBE/2018_vs_pt_tracker_3_workinprogress.pdf}
    \captionof{figure}
        [Scatterplots of the relative lepton \PT error \vs \PT for tracker-driven electrons]
        % TODO: Is it OK to skip over 1.44--1.6?
        {Scatterplots of the relative lepton \PT error \vs \PT for tracker-driven electrons with $0 < \abseta < 0.8$ (left column), $1.0 < \abseta < 1.44$ (middle column), and $1.6 < \abseta < 2.0$ (right column) for 2016 (top row), 2017 (middle row), and 2018 (bottom row) data.}
    \label{fig:2D_Mpas_vs_pt_electron_tracker}
\end{multiFigure}

\subsubsection{Procedure to Derive Corrections to $\delta \ptl$}
% TODO: REWORD
To derive the corrections ($\lambda$), the dilepton invariant mass $\left( \mll \right)$ from \ztolplm events is fitted twice with a Breit--Wigner (BW) function convolved with a Crystal Ball (CB), plus an exponential function (EXP).
In this model, the BW represents the true \mZ lineshape, the CB simulates the detector effects, and the EXP describes the falling background.
When deriving the corrections, the mean and sigma of the BW have been set to the PDG values $\left( \mZ = 91.19\GeV, \; \Gamma_{\PZ} = 2.49\GeV \right)$~\cite{particle_data_group_review_2020}.
 The fit is done in the mass range [60, 120]\GeV, using only
$e^{+}e^{-}$ or $\mu^{+}\mu^{-}$ pairs.\\
The first fit is used to fix all the parameters of the functions but the $\sigma$ of the CB which is
replaced in the second fit by $\lambda$ $\times$ $\delta_{m_{Z}}$, where $\lambda$ is the 
floated parameter of the fit. 

The summary of $\lambda$ correction factors for electrons and muons is shown in Table~\ref{table:Lambdas}.
\begin{table}[!h]\fontsize{10}{12}\selectfont
    \topcaption
        [Event-by-event lambda correction factors.] % TODO: reword
        {Event-by-event lambda correction factors.}
    \begin{tabularx}{\textwidth}{BMSSSSSSSS} % TODO: Fix  numbers in table.
        \toprule 
        \multicolumn{2}{c}{\textbf{2016 pre-VFP}} & \multicolumn{2}{c}{\textbf{2016 post-VFP}} & \multicolumn{2}{c}{\textbf{2017}} & \multicolumn{2}{c}{\textbf{2018}} \\ \toprule

        && \multicolumn{8}{c}{Muons} \\
        \multicolumn{2}{c}{$0 < |\eta| < 0.9$} & 1.05 & 1.248 & 1.043 & 1.214 & 1.047 & 1.183 & 1.019 & 1.227 \\
        \multicolumn{2}{c}{$0.9 < |\eta| < 1.8$} & 1.187 & 1.256 & 1.204 & 1.234 & 1.219 & 1.223 & 1.206 & 1.236 \\
        \multicolumn{2}{c}{$1.8 < |\eta| < 2.4$} & 1.197 & 1.201 & 1.183 & 1.164 & 1.197 & 1.173 & 1.108 & 1.188 \\ \toprule

        && \multicolumn{8}{c}{ECAL electrons} \\ 
        $0 < |\eta| < 0.8$ & $\delta p_T/p_T < 0.01$ & 1.648 & 1.619 & 1.645 & 1.646 & 1.690 & 1.633 & 1.676 & 1.661 \\
        $0 < |\eta| < 0.8$ & $0.01 < \delta p_T/p_T < 0.015$ & 1.490 & 1.506 & 1.470 & 1.528 & 1.540 & 1.496 & 1.544 & 1.511 \\
        $0 < |\eta| < 0.8$ & $0.015 < \delta p_T/p_T < 0.025$ & 1.490 & 1.506 & 1.470 & 1.528 & 1.540 & 1.496 & 1.544 & 1.511 \\
        $0 < |\eta| < 1$ & $0.025 < \delta p_T/p_T < 1$ & 1.490 & 1.506 & 1.470 & 1.528 & 1.540 & 1.496 & 1.544 & 1.511 \\
        $1 < |\eta| < 2.5$ & $\delta p_T/p_T < 0.02$ & 1.490 & 1.506 & 1.470 & 1.528 & 1.540 & 1.496 & 1.544 & 1.511 \\
        $1 < |\eta| < 2.5$ & $0.03 < \delta p_T/p_T < 0.04$ & 1.490 & 1.506 & 1.470 & 1.528 & 1.540 & 1.496 & 1.544 & 1.511 \\
        $1 < |\eta| < 2.5$ & $0.04 < \delta p_T/p_T < 0.05$ & 1.490 & 1.506 & 1.470 & 1.528 & 1.540 & 1.496 & 1.544 & 1.511 \\
        $1 < |\eta| < 2.5$ & $0.05 < \delta p_T/p_T < 0.06$ & 1.490 & 1.506 & 1.470 & 1.528 & 1.540 & 1.496 & 1.544 & 1.511 \\
        $0.8 < |\eta| < 2.5$ & $\delta p_T/p_T < 0.025$ & 1.490 & 1.506 & 1.470 & 1.528 & 1.540 & 1.496 & 1.544 & 1.511 \\
        $1 < |\eta| < 2.5$ & $0.06 < \delta p_T/p_T < 1$ & 1.490 & 1.506 & 1.470 & 1.528 & 1.540 & 1.496 & 1.544 & 1.511 \\ \toprule
        
        && \multicolumn{8}{c}{Tracker electrons} \\ 
        \multicolumn{2}{c}{$0 < |\eta| < 1.44$} & 1.05 & 1.248 & 1.043 & 1.214 & 1.047 & 1.183 & 1.019 & 1.227 \\
        \multicolumn{2}{c}{$1.44 < |\eta| < 1.6$} & 1.187 & 1.256 & 1.204 & 1.234 & 1.219 & 1.223 & 1.206 & 1.236 \\
        \multicolumn{2}{c}{$1.6 < |\eta| < 2$} & 1.197 & 1.201 & 1.183 & 1.164 & 1.197 & 1.173 & 1.108 & 1.188 \\
        \multicolumn{2}{c}{$2 < |\eta| < 2.5$} & 1.197 & 1.201 & 1.183 & 1.164 & 1.197 & 1.173 & 1.108 & 1.188 \\ \hline
    \end{tabularx}
    \label{table:Lambdas}
\end{table}

% \begin{table}[ht]	
% 	\begin{center}
% 		\begin{tabular}{ccccccc}
% 			\hline
% 			\hline			
% 			& \multicolumn{2}{c}{\textbf{2016}} & \multicolumn{2}{c}{\textbf{2017}} & \multicolumn{2}{c}{\textbf{2018}} \\
% 			& \multicolumn{1}{c}{\textbf{MC}} & \multicolumn{1}{c}{\textbf{Data}} & \multicolumn{1}{c}{\textbf{MC}} & \multicolumn{1}{c}{\textbf{Data}} & \multicolumn{1}{c}{\textbf{MC}} & \multicolumn{1}{c}{\textbf{Data}} \\
% 			\hline
% 			\hline
% 			& \multicolumn{6}{c}{Muons} \\
% 			$ 0 < |\eta| < 0.9$ &	1.05	&	1.248	&	1.043	&	1.214	&	1.047		&	1.183	&	1.019	&	1.227	\\
%                        $ 0.9 < |\eta| < 1.8$ &	1.187	&	1.256	&	1.204	&	1.234	&	1.219	&	1.223	&	1.206	&	1.236	\\
%                        $ 1.8 < |\eta| < 2.4$ &	1.197	&	1.201	&	1.183	&	1.164	&	1.197	&	1.173	&	1.108	&	1.188	\\
% 			\hline
% 			\hline
% 			& \multicolumn{6}{c}{ECAL electrons} \\
% 			$0 < |\eta| < 0.8$ and $\delta\pt/\pt < 0.01$            & 2.006 & 1.893 & 2.086 & 2.030 & 2.054 & 1.914 \\
%                        $0 < |\eta| < 0.8$ and $0.01 < \delta\pt/\pt < 0.015$    & 1.590 & 1.575 & 1.698 & 1.680 & 1.701 & 1.635\\
%                        $0 < |\eta| < 0.8$ and $0.015 < \delta\pt/\pt < 0.025$   & 1.406 & 1.373 & 1.426 & 1.450 & 1.447 & 1.467\\
%                        $0 < |\eta| < 1$ and $0.025 < \delta\pt/\pt < 1$         & 1.517 & 1.531 & 1.481 & 1.521 & 1.560 & 1.569\\
%                        $1 < |\eta| < 2.5$ and $\delta\pt/\pt < 0.02$            & 2.116 & 2.002 & 2.305 & 2.210 & 2.324 & 2.228\\
%                        $1 < |\eta| < 2.5$ and $0.02 < \delta\pt/\pt < 0.03$     & 1.645 & 1.623 & 1.815 & 1.795 & 1.787 & 1.759\\
%                        $1 < |\eta| < 2.5$ and $0.03 < \delta\pt/\pt < 0.04$     & 1.472 & 1.489 & 1.568 & 1.560 & 1.468 & 1.509\\
%                        $1 < |\eta| < 2.5$ and $0.04 < \delta\pt/\pt < 0.06$     & 1.374 & 1.448 & 1.414 & 1.606 & 1.378 & 1.477\\
%                        $0.8 < |\eta| < 1$ and $\delta\pt/\pt < 0.025$           & 1.149 & 1.203 & 1.196 & 1.241 & 1.180 & 1.286\\
%                        $1 < |\eta| < 2.5$ and $0.06 < \delta\pt/\pt < 1$        & 1.099 & 1.221 & 1.171 & 1.331 & 1.123 & 1.272\\
% 			\hline
% 			\hline
% 			& \multicolumn{6}{c}{tracker electrons} \\
% 			$0 < |\eta| < 1.44$ & 1.619 & 1.872 & 2.382 & 2.115 & 2.120 & 1.936\\
%                        $1.44 < |\eta| < 1.6$ & 6.452 & 5.900 & 6.572 & 7.056 & 5.613 & 5.524\\
%                        $1.6 < |\eta| < 2$ & 2.732 & 2.826 & 3.430 &2.846& 3.204 & 3.016\\
%                        $2 < |\eta| <2.5$ & 3.010 &3.081& 3.963 &3.817& 4.110 & 3.762\\
% 			\hline
% 		\end{tabular}
% 	\end{center}
% 	\caption{\PT error corrections for muons and electrons in different kinematic region. For each year, MC 
% 	is on the left, data on the right.}
% 	\label{table:Lambdas}
% \end{table}
% TODO: Figure 8 (from AN) of lambdas vs. dpT/pT bins? CONTAINS DATA.

% TODO: REWORD.
\subsubsection{Validation of corrections (MC, data)}
\label{sec:ClosureTest}
A closure test is performed to validate correction derived for lepton \PT error. \\
First, events are divided according to different predicted $\delta m_{Z}/m_{Z}$ ranges before corection. 
Then, in each bin, the dilepton invariant mass distribution is fitted using a BW convoluted with CB 
plus exponential function, to get $\delta m_{Z}^{fit}$ (measured $m_{Z}$ resolution). Finally 
the average predicted $\delta m_{Z}$ is calculated in each $\delta m_{Z}/m_{Z}$ bin before and after the 
correction factor for lepton \PT error is applied (predicted $m_{Z}$ resolution). \\
In the closure plot, it is expected to see $\delta m_{Z}$ gets closer to $\delta m_{Z}^{fit}$ 
after correction, and the points should stay in a band which is 20\% around diagonal line, which 
is the uncertainty assigned to the resolution in the previous analysis \cite{HIG_16_041}. 
This closure test is shown in Fig.~\ref{fig:ZClosure_test_MU} for muons and in Fig.~\ref{fig:ZClosure_test_ELE} 
for electrons. A further check has been also performed, looking at the closure test of the predicted four lepton mass 
resolution compared to the fitted four lepton mass resolution using H signal MC samples once the 
corrections derived using \PZ events are applied. After applying corrections, measured $m_{4l}$ 
resolution gets closer to the prediction. This closure test is shown for three different 
final states in Fig.~\ref{fig:HClosure_test}.
% CLOSURE: Z->mumu MC.
\begin{multiFigure}
    \centering
    \addFigure{0.45}{../../higgsmassmeasurement/AN-19-248/Figures/EBE/VXBS_20160_MC_workinprogress.pdf}
    \addFigure{0.45}{../../higgsmassmeasurement/AN-19-248/Figures/EBE/VXBS_20165_MC_workinprogress.pdf}
    \addFigure{0.45}{../../higgsmassmeasurement/AN-19-248/Figures/EBE/VXBS_2017_MC_workinprogress.pdf}
    \addFigure{0.45}{../../higgsmassmeasurement/AN-19-248/Figures/EBE/VXBS_2018_MC_workinprogress.pdf}
    \captionof{figure}
        % [Closure plots showing corrections to lepton momentum uncertainty on \ztomumu events]
        [Validation of the per-event mass uncertainties from simulated \ztomumu events]
        {Validation of the per-event mass uncertainties from simulated \ztomumu events.
        The solid blue line corresponds to a 10\% band relative to the solid green 1:1 line.
        \;A) 2016 pre-VFP.
        \;B) 2016 post-VFP.
        \;C) 2017.
        \;D) 2018.
		}
    \label{fig:ZClosure_test_MU}
\end{multiFigure}
% CLOSURE: Z->ee MC.
\begin{multiFigure}
    \centering
    \addFigure{0.45}{../../higgsmassmeasurement/AN-19-248/Figures/EBE/ZEE_20160_MC_workinprogress.pdf}
    \addFigure{0.45}{../../higgsmassmeasurement/AN-19-248/Figures/EBE/ZEE_20165_MC_workinprogress.pdf}
    \addFigure{0.45}{../../higgsmassmeasurement/AN-19-248/Figures/EBE/ZEE_2017_MC_workinprogress.pdf}
    \addFigure{0.45}{../../higgsmassmeasurement/AN-19-248/Figures/EBE/ZEE_2018_MC_workinprogress.pdf}
    \captionof{figure}
        [Validation of the per-event mass uncertainties using simulated \ztoee events]
        {Validation of the per-event mass uncertainties using simulated \ztoee events.
        The solid blue line corresponds to a 10\% band relative to the solid green 1:1 line.
        \;A) 2016 pre-VFP.
        \;B) 2016 post-VFP.
        \;C) 2017.
        \;D) 2018.
        }
    \label{fig:ZClosure_test_ELE}
\end{multiFigure}
% CLOSURE: ggH4l MC.
% $\sigma_\text{m}^\text{2e2\Pmu}\;(\GeVns)$ % FOR TESTING. DELETE
\begin{multiFigure}
    \centering
    \addFigure{0.32}{../../higgsmassmeasurement/AN-19-248/Figures/EBE/2016_H4mu_Closure_test_workinprogress.png}
    \addFigure{0.32}{../../higgsmassmeasurement/AN-19-248/Figures/EBE/2016_H4e_Closure_test_workinprogress.png}
    \addFigure{0.32}{../../higgsmassmeasurement/AN-19-248/Figures/EBE/2016_H2e2mu_Closure_test_workinprogress.png}

    \addFigure{0.32}{../../higgsmassmeasurement/AN-19-248/Figures/EBE/2017_H4mu_Closure_test_workinprogress.png}
    \addFigure{0.32}{../../higgsmassmeasurement/AN-19-248/Figures/EBE/2017_H4e_Closure_test_workinprogress.png}
    \addFigure{0.32}{../../higgsmassmeasurement/AN-19-248/Figures/EBE/2017_H2e2mu_Closure_test_workinprogress.png}
    
    \addFigure{0.32}{../../higgsmassmeasurement/AN-19-248/Figures/EBE/2018_H4mu_Closure_test_workinprogress.png}
    \addFigure{0.32}{../../higgsmassmeasurement/AN-19-248/Figures/EBE/2018_H4e_Closure_test_workinprogress.png}
    \addFigure{0.32}{../../higgsmassmeasurement/AN-19-248/Figures/EBE/2018_H2e2mu_Closure_test_workinprogress.png}
    \captionof{figure}
        [Validation of the per-event mass uncertainties using simulated \gghtofourl events]
        {Validation of the per-event mass uncertainties using simulated \gghtofourl events for
        2016 (top row), 2017 (middle row), and 2018 (bottom row) in three different final states:
		\fourmu (left column), \foure (middle column), and \twoetwomu (right column).
		The 20\% reference band is also shown.}
    \label{fig:HClosure_test}
\end{multiFigure}

\subsection{Vertex+Beamspot Constraint: VXBS}
Leptons produced from the \hzzfourl channel should originate from the \pp collision point---the primary vertex---since the \PH boson and both \PZ bosons decay promptly after being formed.
The primary vertex from which the muons originate can be approximated to come from the more general \emph{beamspot}, \ie the luminous region in the $x$-$y$ plane where the \pp bunches cross.
Thus, if the muon tracks are constrained to come from their vertex of origination ($\sim$beamspot), then one should expect a more precise measurement of muon momentum and resolution.
The updated muon kinematical variables then get propagated into a more precise estimate of the \mH value per event.
This process of constraining muon tracks to a vertex that is compatible with the beamspot is called the `vertex+beamspot constraint' (VXBS).

The Phase-1 pixel upgrade during the middle of Run 2 allowed for a more collimated (\emph{more precise}) \pp beamspot.
\Cref{fig:BeamY_vs_Y} shows the improvement on a tighter beamspot when moving from 2016 Run G to 2018 Run D.
Both the $x$ and $y$ widths of the beamspot ($\sigma_x$ and $\sigma_y$, respectively) are smaller:
\begin{align*}
    \sigma_x^{2016} \sim 14\mum \quad &  \vs \quad \sigma_x^{2018} \sim 9\mum
    \\
    \vspace{3pt}
    \sigma_y^{2016} \sim 9\mum \quad &  \vs \quad \sigma_y^{2018} \sim 7\mum,
\end{align*}
giving a better vertex constraint for muons when it is used in the track reconstruction.
Once the event is selected, the vertex+beamspot constraint is applied.
%%%%%%%%%%%%%%%%%%%%%%%%%%%%%%%
% Beamspot width Y \vs. width X.
\begin{multiFigure}
    \centering
        \addFigure{0.48}{figures/higgsmassmeas/vxbs/2016G_VS_width_y_vs_x_workinprogress.png}
        \addFigure{0.48}{figures/higgsmassmeas/vxbs/2018D_VS_width_y_vs_x_workinprogress.png}
	\captionof{figure}
        [Two-dimensional histogram of beamspot widths, $\sigma_y$ \vs $\sigma_x$, comparing 2016 Run G and 2018 Run D]
        {Two-dimensional histogram of beamspot widths (in \cmns), $\sigma_y$ \vs $\sigma_x$ for different runs.
        The 2018 runs showed narrower beamspot spreads, which assists with the vertex+beamspot constraint.
        \;A) For 2016 Run G.
        \;B) For 2018 Run D.}
    \label{fig:BeamY_vs_Y}
\end{multiFigure}
%%%%%%%%%%%%%%%%%%%%%%%%%%%%%%%

The muon reconstruction that is currently used does not take into account any information about the vertex from which the lepton originates. % and in the Higgs boson decay, all the leptons are prompt.
In the VXBS approach, the four muon tracks from a Higgs boson decay are constrained to a common vertex (VX) that must be compatible with the beamspot (BS). 
So while the kinematical information of muons is updated after this procedure, the information of electrons is left unaltered, as they are only used to help constrain the position of the common vertex.

This method has been checked in data and simulation using \ztomumu events, looking in the \mZ range of 60--120\GeV.
The improvement of muon momentum resolution is shown in \cref{fig:Resolution_pT,fig:Resolution_eta}, as a function of \pT and $|\eta|$, respectively, for 2016--2018.
It was seen that \pT resolution and \mZ resolution improves by about 5--10\%, depending on the year.
\begin{multiFigure}
    \centering
        \addFigure{0.48}{figures/higgsmassmeas/vxbs/Resolution_2016_preVFP_PT_workinprogress.png}
        \addFigure{0.48}{figures/higgsmassmeas/vxbs/Resolution_2016_postVFP_PT_workinprogress.png}
        \addFigure{0.48}{figures/higgsmassmeas/vxbs/Resolution_2017_PT_workinprogress.png}
        \addFigure{0.48}{figures/higgsmassmeas/vxbs/Resolution_2018_PT_workinprogress.png}
    \captionof{figure}
        [Muon \pT resolution as a function of \pT before and after VXBS constraint]
        {Muon \pT resolution as a function of \pT before (black line) and after (red line) VXBS constraint using \ztomumu events in data.
        The ratio plot at the bottom of each subfigure shows improvement in muon \pT resolution, with improvement as high as approximately 15\% in 2018 events.
        \;A) For 2016 pre-VFP.
        \;B) For 2016 post-VFP.
        \;C) For 2017.
        \;D) For 2018.}
    \label{fig:Resolution_pT}
\end{multiFigure}
\begin{multiFigure}
    \centering
        \addFigure{0.48}{figures/higgsmassmeas/vxbs/Resolution_2016_preVFP_ETA_workinprogress.png}
        \addFigure{0.48}{figures/higgsmassmeas/vxbs/Resolution_2016_postVFP_ETA_workinprogress.png}
        \addFigure{0.48}{figures/higgsmassmeas/vxbs/Resolution_2017_ETA_workinprogress.png}
        \addFigure{0.48}{figures/higgsmassmeas/vxbs/Resolution_2018_ETA_workinprogress.png}
    \captionof{figure}
        [Muon \pT resolution as a function of \abseta before and after VXBS constraint]
        {Muon \pT resolution as a function of \abseta before (black line) and after (red line) VXBS constraint using \ztomumu events in data.
        The ratio plot at the bottom of each subfigure shows improvement in muon \pT resolution, with improvement as high as approximately 10\% for all years.
        \;A) For 2016 pre-VFP.
        \;B) For 2016 post-VFP.
        \;C) For 2017.
        \;D) For 2018.}
\label{fig:Resolution_eta}
\end{multiFigure}

\Cref{UL_ZBoson_DataMC_comparison_VXBS} shows Data--MC comparison, with and without VXBS constraint.
The mean and $\sigma$ reported on the plots belong to the DSCB---which is convolved with a BW---and fit to the distribution.
As can been seen, the resolution of \mll gets improved comparably to that of the improvement of muon \pT resolution:
5\% (5\%) for 2016 pre-VFP,
5\% (6\%) for post-VFP,
5\% (4\%) for 2017, and
7\% (8\%) for 2018 Data (MC).
\begin{multiFigure}
    \centering
        \includegraphics[width=0.96\textwidth]{figures/higgsmassmeas/vxbs/vxbs_mZdist_2017_2018.png}
    \captionof{figure}
        [Distributions of \mll using data and MC, with and without the VXBS constraint]
        {Distributions of \mll using data and MC, with (left column) and without (right column) the VXBS constraint. % TODO: REWORD 
        \;A) 2018.
        \;B) 2017.
        \;C) 2016 post-VFP.
        \;D) 2016 pre-VFP.}
    \label{UL_ZBoson_DataMC_comparison_VXBS}
\end{multiFigure}

%=== 1D likelihood fit after VXBS ===%
The updated invariant mass distributions, after applying the VXBS constraint, are shown in Fig.~\ref{fig:1D_VXBS_mass_2018_ggH}.
\begin{multiFigure}
    \centering
        \addFigure{0.45}{figures/higgsmassmeas/ggH_MassDistribution/1D_VXBS_mass_2018_ggH_4mu.pdf}
        \addFigure{0.45}{figures/higgsmassmeas/ggH_MassDistribution/1D_VXBS_mass_2018_ggH_4e.pdf}
        \addFigure{0.45}{figures/higgsmassmeas/ggH_MassDistribution/1D_VXBS_mass_2018_ggH_2e2mu.pdf}
        \addFigure{0.45}{figures/higgsmassmeas/ggH_MassDistribution/1D_VXBS_mass_2018_ggH_2mu2e.pdf}
    \captionof{figure}
        [Four-lepton invariant mass distribution, after applying the VXBS constraint]
        {Four-lepton invariant mass distribution, after applying the VXBS constraint in the signal region ([105-140]\GeV) using ggH events, with the DSCB fit for 2018.
        \;A) \fourmu.    % TODO: fix symbols
        \;B) \foure.
        \;B) \twoetwomu.
        \;B) \twomutwoe.}
    \label{fig:1D_VXBS_mass_2018_ggH}
\end{multiFigure}
As can been seen, comparing these new distributions (\cref{fig:1D_VXBS_mass_2018_ggH}) with the ones before beamspot constraint (\cref{fig:1D_VXBS_mass_2018_ggH}), the $\sigma$ in \fourmu final state has improved by 7$\%$, \foure is not affected (as expected), while mixed flavor final states show an improvement of a few percent.

\subsubsection{Expected \mH measurement uncertainties (MC)}
The expected $\mass{\PH}$ measurement uncertainties comparing VXBS constraint to the baseline expectations can be seen in \cref{table:1D_model_result_fs_BS} split by final state or in \cref{table:1D_model_result_year_BS} split by year.

\begin{table}[ht]	
\begin{center}
    \captionof{table}
        [Expected Higgs boson mass uncertainty measured with 1D model, with and without VXBS, by final state]
        {Expected Higgs boson mass uncertainty measured with 1D model, with and without
        VXBS for different final states. All mass values are given in \MeVns.
        Statistical-only results are considered at this stage of the analysis.}
    \begin{tabular}{ccccccc}
            \hline			
        Expected uncertainty (\MeVns)	&	\fourmu	&	\foure	&	\twoetwomu	&\twomutwoe	& inclusive	& Rel. Improvement \\
            \hline			
        1$D_\text{VXBS}$  (No bkg)	&	137	&	394	&	275	&	266	&	108 &	$-$4\%	\\
            1D	(No bkg) &	147	&	394	&	276	&	273	&	112	&	---	\\
        %	1$D_\text{VXBS}$  (No bkg)	&	144	&	466	&	313	&	291	&	116	&	-4\%	\\	
        %	1D	(No bkg) &	153	&	466	&	315	&	300	&	121 	&	- \\
            \hline
    \end{tabular}
    \label{table:1D_model_result_fs_BS}
\end{center}
\end{table}
\begin{table}[ht]	
\begin{center}
    \captionof{table}
        [Expected Higgs boson mass uncertainty measured with 1D model, with and without VXBS, by year]
        {Expected Higgs boson mass uncertainty measured with 1D model, with and without VXBS for different years. All mass values are given in \MeVns.
        Statistical-only results are considered at this stage of the analysis.}
    \begin{tabular}{ccccc}
        \hline			
    Expected uncertainty (\MeVns)	&	2016 pre-VFP	&	2016 post-VFP	&	2017	&	2018	\\
        \hline			
        1$D_\text{VXBS}$	(No bkg)	&	291	&	301	&	198	&	162	\\
        1D (No bkg)	&	304	&	314	&	207	&	170	\\
        \hline
    \end{tabular}
    \label{table:1D_model_result_year_BS}
\end{center}
\end{table}

\subsubsection{Alternative to VXBS: The Ad Hoc Method}
It was seen that the lepton \pT resolution depends upon the product of lepton charge $(q)$ and lepton impact parameter $(d_0)$.
See Appendix~\ref{app:adhoc_studies} for a motivation of this study and derivation of the formula for $d_0$.

% \subsection{\texorpdfstring{\Zone}{Z1} Mass Constraint}  % TODO: Makes Z1 in subsection title NOT bold.
\subsection{Refitting muon and electron \pt with a \texorpdfstring{\Zone}{Z1} Mass Constraint}  % TODO: Makes Z1 in subsection title NOT bold.
% % \subsection{\texorpdfstring{\boldmath{\Zone}}{Z1} Mass Constraint} % TODO: Although \boldmath makes \subsection bold, it also makes the TOC entry bold.
\label{sec:Z1constraint}

%\subsubsection{Z1-mass line shape}
% \subsubsection{Methodology}
% TODO:REWORD, USE CONSISTENT SYMBOLS.
In order to improve the four lepton invariant mass resolution, a kinematic fit is also performed
using a mass constraint on the intermediate on-shell Z resonance, using an approach similar to the one
described previously.
% TODO:CITE
% \cite{RefToZrefit}.
The basic idea is to re-evaluate \pT of two leptons forming 
the Z bosons of the Higgs candidate, with a constraint on the reconstructed Z mass to follow 
the Z boson true lineshape. For a 125 \GeV Higgs, the selected $Z_{1}$ is mostly on-shell, 
while $m_{Z_{2}}$ distribution is broad and the spread is much bigger than detector resolution. 
When considering mass measurment of 125 \GeV Higgs, expected gain in resolution comes from refitting $Z_{1}$.
The likelihood to be maximized can be written as:
\[
\mathcal{L}(p_{T}^{1} , p_{T}^{2}|p_{T}^{reco1}, \sigma p_{T}^{1},p_{T}^{reco2}, \sigma p_{T}^{2}) = 
Gauss(p_{T}^{reco1}|p_{T}^{1}, \sigma p_{T}^{1}) \cdot Gauss(p_{T}^{reco2}|p_{T}^{2}, \sigma p_{T}^{2}) 
\cdot \mathcal{L}(m_{12}|m_{Z},m_{H})
\]
where $p_{T}^{reco1,2}$ are the reconstructed transverse momentum of the two leptons forming the $Z_{1}$,
$\sigma_{p_{T}^{1,2}}$ are the per lepton resolution (uncertainty on \pT measurement, corrected using 
method described in \ref{sec:ebe}), $p_{T}^{1,2}$ are the observables under optimisation,
$m_{12}$ is the invariant mass calculated from $p_{T}^{1}$ and $p_{T}^{2}$. $\mathcal{L}(m_{12}|m_{Z},m_{H})$
is the likelihood, given the true lineshape of $m_{Z_{1}}$. \\
%For a 125 \GeV Higgs, the selected $Z_{1}$ is not always on-shell, so the Breit Wigner shape can not describe
%perfectly its lineshape at generator level. In principle, one can choose any of the $Z_{1}$ 
%lineshapes to be used in the refitting procedure. As long as all Monte Carlor samples and data
%events go through the same procedure (including the same true $Z_{1}$ lineshape), 
%no bias would be introduced.  We choose true gen level $Z_{1}$ lineshape from the SM Higgs boson sample to
%optimize the sensitivity.
For each event, the likelihood is maximized and \pT information of the refitted leptons are updated. \\
Fig.~\ref{fig:Z1_lineshape} shows the on-shell Z resonance line shape, at generator level, 
used to perform the kinematical fit, as taken from ggH 125 \GeV sample. 
Events from all three years have been merged, asking for a 
$p_T > 20(10)$\GeV for (sub)leading leptons, and $|\eta|<$2.4(2.5) for $\mu$(e), focusing on
the reconstructed mass range of [105-140] \GeV in $m_{4l}^{RECO}$. The fit function used is a convolution
of a CB plus three different guassians.
\begin{multiFigure}
    \centering
        \addFigure{0.45}{../../higgsmassmeasurement/AN-19-248/Figures/Z1_lineshape/4mu.pdf}
        \addFigure{0.45}{../../higgsmassmeasurement/AN-19-248/Figures/Z1_lineshape/4e.pdf}
        \addFigure{0.45}{../../higgsmassmeasurement/AN-19-248/Figures/Z1_lineshape/2e2mu.pdf}
        \addFigure{0.45}{../../higgsmassmeasurement/AN-19-248/Figures/Z1_lineshape/2mu2e.pdf}
	\captionof{figure}
        [On-shell Z resonance line shape at generator level, as taken from ggH sample @ 125\GeV]
        {On-shell Z resonance line shape at generator level, as taken from ggH sample @ 125\GeV, % TODO:REWORD
        merging all three years for the 4 different final states.
        \;A) 4$\mu$.
        \;B) 4e.
        \;C) 2e2$\mu$.
        \;D) 2$\mu$2e.}
    \label{fig:Z1_lineshape}
\end{multiFigure}
%After this kinimatic refitting, the mass of the Z candidate, the $m_{4\ell}$, and the $\sigma_{m_{4\ell}}$
%are recalculated. The comparison of the reconstructed and refitted mass can be seen in Fig.~\ref{Z1constraint}.

% \subsection{Mass distributions}
The new invariant mass distributions, with also the constraint of the on-shell Z1 boson, are shown in 
Fig.~\ref{fig:1D_VXBS_Z1_mass_2018_ggH}.
\begin{multiFigure}
    \centering
        \addFigure{0.45}{figures/higgsmassmeas/ggH_MassDistribution/1D_VXBS_Z1_mass_2018_ggH_4mu.pdf}
        \addFigure{0.45}{figures/higgsmassmeas/ggH_MassDistribution/1D_VXBS_Z1_mass_2018_ggH_4e.pdf}
        \addFigure{0.45}{figures/higgsmassmeas/ggH_MassDistribution/1D_VXBS_Z1_mass_2018_ggH_2e2mu.pdf}
        \addFigure{0.45}{figures/higgsmassmeas/ggH_MassDistribution/1D_VXBS_Z1_mass_2018_ggH_2mu2e.pdf}
    \captionof{figure}
        [Four-lepton invariant mass distribution, with VXBS and Z1 constraint]
        {Four-lepton invariant mass distribution, with VXBS and Z1 constraint, in the signal region ([105-140]\GeV) using ggH events, with the DSCB fit for 2018: % TODO:REWORD
        \;A) 4$\mu$.
        \;B) 4e.
        \;C) 2e2$\mu$.
        \;D) 2$\mu$2e.} 
    \label{fig:1D_VXBS_Z1_mass_2018_ggH}
\end{multiFigure}
Comparing these new distributions (Fig.~\ref{fig:1D_VXBS_Z1_mass_2018_ggH}) with the previous ones (Fig.~\ref{fig:1D_VXBS_mass_2018_ggH}), it can be seen that the \Zone constraint has a bigger improvement in final states with on-shell Z boson decaying in to 2e (~30$\%$ improvement on $\sigma$ for 2e2$\mu$ and 16$\%$ for 4e) while the other two final states (4$\mu$ and 2$\mu$2e) are less affected (improvement in $\sigma$ of the other of 5$\%$).

\subsubsection{Expected mH measurement uncertainties (MC) and relative improvements}
The new four-lepton mass ($m'_{4\ell}$), shown in Fig.~\ref{fig:1D_VXBS_Z1_mass_2018_ggH},
is used to rebuild the 1D likelihood function, $\mathcal{L}$($m'_{4\ell}|m_{H}$). Signal normalisation and signal parameterization are extracted following the procedure described in \ref{sec:signal_model}.
The expected \mH measurement uncertainty, is reported in Table~\ref{table:1D_model_result_Z1} (or in Table~\ref{table:1D_model_result_Z1_year}
splitted in years).
%===
\begin{table}[ht]	
\begin{center}
    \topcaption
        [Expected Higgs boson mass uncertainty measured with 1D model, with and without
        Z1 refit for different final states]
        {Expected Higgs boson mass uncertainty measured with 1D model, with and without
        Z1 refit for different final states. All mass values are given in \MeV.
        Statistical only results are considered at this stage of the analysis.
        }
    \begin{tabular}{ccccccc}
        \hline			
    Expected uncertainty	&	4$\mu$	&	4e	&	2e2$\mu$	&2$\mu$2e	& inclusive & Rel. Improvement \\
        \hline			
        1$D'_{VXBS}$ (No bkg)	&	129	&	357	&	223	&	255	&	99	&	-8\%	\\
        1$D_{VXBS}$  (No bkg)	&	137	&	394	&	275	&	266	&	108 &	-4\%	\\
    %	1$D'_{VXBS}$ (No bkg)	&	134	&	424	&	250	&	279	&	105	&	-9\%	\\
    %	1$D_{VXBS}$  (No bkg)	&	144	&	466	&	313	&	291	&	116	&	-\\	
        %\hline
    %relative improvement	&	-	&	-	&	-	&	-	&	-	\\
        \hline
\end{tabular}
\label{table:1D_model_result_Z1}
\end{center}
\end{table}
%===
\begin{table}[ht]	
\begin{center}
    \topcaption
        [Expected Higgs boson mass uncertainty measured with 1D model, with and without
        Z1 refit for different years]
        {Expected Higgs boson mass uncertainty measured with 1D model, with and without
        Z1 refit for different years. All mass values are given in \MeV.
        Statistical only results are considered at this stage of the analysis.
        }
    \begin{tabular}{ccccc}
        \hline			
    Expected uncertainty	&	2016 pre-VFP	&	2016 post-VFP	&	2017	&	2018	\\
        \hline			
        1$D'_{VXBS}$ (No bkg)	&	266	&	276	&	182	&	148	\\
        1$D_{VXBS}$	(No bkg)	&	291	&	301	&	198	&	162	\\
    %	1$D'_{VXBS}$ (No bkg)	&	206	&	194	&	157	\\
    %	1$D_{VXBS}$	(No bkg)	&	227	&	212	&	173	\\
        \hline
    \end{tabular}
    \label{table:1D_model_result_Z1_year}
\end{center}
\end{table}

% Expected mH measurement uncertainties using 3D pdf(m4l, sigma, Dkin | mH)
% with background (no systematics yet)
%\section{$D_{m_{4\ell}}$ categorization}
\subsection{Relative Mass Error Categorization}
\label{sec:SignalParam_N_2D}
Previously, a 3D likelihood fit has been used to extract final Higgs boson mass uncertainty. 
For the full Run 2 results, a categorization based on the relative mass error is implemented.
This method will help, not only in taking into account its correlation with \Dkinbkg (see \ref{sec:DkinCorrelation}), but it will also improve the signal parameterization.
In particular, the signal parameters of the DSCB depend on both the mass and the relative mass error.

Starting from the raw distribution of the relative mass error and a simulated ggH sample with $\mH \sim 125\GeV$, 
the distribution is divided into 9 equal-entry bins (\ie in order to guarantee an equal number of raw events in each bin).
Bin spliting is done separately for each year and for each final state.
The procedure is demonstrated in \cref{Bin_splitting_2018} for 2018 MC.
\begin{multiFigure}
    \centering
        \addFigure{0.48}{../../higgsmassmeasurement/AN-19-248/Figures/Bin_splitting/RelMassErroSplitting_4mu.pdf}
        \addFigure{0.48}{../../higgsmassmeasurement/AN-19-248/Figures/Bin_splitting/RelMassErroSplitting_4e.pdf}
        \addFigure{0.48}{../../higgsmassmeasurement/AN-19-248/Figures/Bin_splitting/RelMassErroSplitting_2e2mu.pdf}
        \addFigure{0.48}{../../higgsmassmeasurement/AN-19-248/Figures/Bin_splitting/RelMassErroSplitting_2mu2e.pdf}
    \captionof{figure}
        [Distribution of the relative mass error for a 2018 ggH sample, where $\mH \sim 125\GeV$]
        {Distribution of the relative mass error for a 2018 ggH sample, where $\mH \sim 125\GeV$.
        The dotted black lines show the equal-entry bin edges used for the categorization.
        \;A) \fourmu final state.
        \;B) \foure final state.
        \;C) \twoetwomu final state.
        \;D) \twomutwoe final state.}
    \label{Bin_splitting_2018}
\end{multiFigure}

To extract the mass measurement using this categorization method,
the signal is modeled independently in each bin, 
yielding individual (semi)final measurements which are then combined in the data cards.

Examples of the signal parameterization for 2018 samples are shown in \cref{signal_lineshape_2018_4mu,signal_lineshape_2018_4e,signal_lineshape_2018_2e2mu,signal_lineshape_2018_2mu2e}. 
Each figure shows the fit of the 125\GeV ggH sample (in blue, top left) and the simultaneous fits of the all five mass points (red scale) in few random bins.
% TODO: Reword
\begin{multiFigure}
    \centering
        \addFigure{0.48}{../../higgsmassmeasurement/AN-19-248/Figures/Categorisation/DSCB_4mu_ggF_2018_2.pdf}
        \addFigure{0.48}{../../higgsmassmeasurement/AN-19-248/Figures/Categorisation/DSCB_4mu_ggF_2018_3.pdf}
        \addFigure{0.48}{../../higgsmassmeasurement/AN-19-248/Figures/Categorisation/DSCB_4mu_ggF_2018_8.pdf}
        \addFigure{0.48}{../../higgsmassmeasurement/AN-19-248/Figures/Categorisation/DSCB_4mu_ggF_2018_9.pdf}
    \captionof{figure}
        [Fit examples for four $D'^{\text{VXBS}}_{m_{4\ell}}$ bins (2nd, 3rd, 8th, and 9th) for \fourmu 2018]
        {Fit examples for four $D'^{\text{VXBS}}_{m_{4\ell}}$ bins (2nd, 3rd, 8th, and 9th) for \fourmu 2018.
        Top left: 125\GeV sample only fit. The others plots stand for the simultaneous fit of all the mass points:
        starting from top right respectively 125, 120, 124, 126, 130\GeV.
        \;A) 2nd bin.
        \;B) 3rd bin.
        \;C) 8th bin.
        \;D) 9th bin.}
\label{signal_lineshape_2018_4mu}
\end{multiFigure}
% TODO: Reword
\begin{multiFigure}
    \centering
        \addFigure{0.48}{../../higgsmassmeasurement/AN-19-248/Figures/Categorisation/DSCB_4e_ggF_2018_1.pdf}
        \addFigure{0.48}{../../higgsmassmeasurement/AN-19-248/Figures/Categorisation/DSCB_4e_ggF_2018_2.pdf}
        \addFigure{0.48}{../../higgsmassmeasurement/AN-19-248/Figures/Categorisation/DSCB_4e_ggF_2018_5.pdf}
        \addFigure{0.48}{../../higgsmassmeasurement/AN-19-248/Figures/Categorisation/DSCB_4e_ggF_2018_9.pdf}
    \captionof{figure}
        [Fit examples for four $D'^{\text{VXBS}}_{m_{4\ell}}$ bins (1st, 2nd, 5th, and 9th) for \foure 2018]
        {Fit examples for four $D'^{\text{VXBS}}_{m_{4\ell}}$ bins (1st, 2nd, 5th, and 9th) 
        for \foure 2018. Top left: 125 GeV sample only fit. The others plots stand for the simultaneous 
        fit of all the mass points: starting from top right respectively 125, 120, 124, 126, 130\GeV.
        \;A) 1st bin.
        \;B) 2nd bin.
        \;C) 5th bin.
        \;D) 9th bin.}
    \label{signal_lineshape_2018_4e}
\end{multiFigure}
% TODO: Reword
\begin{multiFigure}
    \centering
        \addFigure{0.48}{../../higgsmassmeasurement/AN-19-248/Figures/Categorisation/DSCB_2e2mu_ggF_2018_1.pdf}
        \addFigure{0.48}{../../higgsmassmeasurement/AN-19-248/Figures/Categorisation/DSCB_2e2mu_ggF_2018_3.pdf}
        \addFigure{0.48}{../../higgsmassmeasurement/AN-19-248/Figures/Categorisation/DSCB_2e2mu_ggF_2018_7.pdf}
        \addFigure{0.48}{../../higgsmassmeasurement/AN-19-248/Figures/Categorisation/DSCB_2e2mu_ggF_2018_9.pdf}
    \captionof{figure}
        [Fit examples for four $D'^{\text{VXBS}}_{m_{4\ell}}$ bins (1st, 3rd, 7th, and 9th) for \twoetwomu 2018]
        {Fit examples for four $D'^{\text{VXBS}}_{m_{4\ell}}$ bins (1st, 3rd, 7th, and 9th) 
        for \twoetwomu 2018. Top left: 125 GeV sample only fit. The others plots stand for the simultaneous 
        fit of all the mass points: starting from top right respectively 125, 120, 124, 126, 130 GeV.
        \;A) 1st bin.
        \;B) 3rd bin.
        \;C) 7th bin.
        \;D) 9th bin.}
        \label{signal_lineshape_2018_2e2mu}
\end{multiFigure}
% TODO: Reword
\begin{multiFigure}
    \centering
        \addFigure{0.48}{../../higgsmassmeasurement/AN-19-248/Figures/Categorisation/DSCB_2mu2e_ggF_2018_1.pdf}
        \addFigure{0.48}{../../higgsmassmeasurement/AN-19-248/Figures/Categorisation/DSCB_2mu2e_ggF_2018_4.pdf}
        \addFigure{0.48}{../../higgsmassmeasurement/AN-19-248/Figures/Categorisation/DSCB_2mu2e_ggF_2018_6.pdf}
        \addFigure{0.48}{../../higgsmassmeasurement/AN-19-248/Figures/Categorisation/DSCB_2mu2e_ggF_2018_9.pdf}
    \captionof{figure}
        [Fit examples for four $D'^{\text{VXBS}}_{m_{4\ell}}$ bins (1st, 4th, 6th, and 9th) 
        for \twomutwoe 2018]
        {
            % Fit examples for four $D'^{\text{VXBS}}_{m_{4\ell}}$ bins (1st, 4th, 6th, and 9th) 
        for \twomutwoe 2018. Top left: 125 GeV sample only fit. The others plots stand for the simultaneous 
        fit of all the mass points: starting from top right respectively 125, 120, 124, 126, 130 GeV.
        \;A) 1st bin.
        \;B) 4th bin.
        \;C) 6th bin.
        \;D) 9th bin.} % TODO: double check bin numbers.
    \label{signal_lineshape_2018_2mu2e}
\end{multiFigure}
Looking in each bin at the parameter mean of the DSCB function, used to fit the 125\GeV sample, 
(\cref{MeanDependence}), it can be seen that if in \fourmu final state (black dots) the mean
is quite stable, in the \foure (in red) and \twoetwomu (green) final states, it shifts towards lower values
($\mathcal{O}$({GeV})).
Categorization  helps to properly describe the Higgs boson line shape, not only as a function of mass, but also as a function of mass resolution.
% TODO: Reword
\begin{multiFigure}
    \centering
        \addFigure{0.48}{../../higgsmassmeasurement/AN-19-248/Figures/Categorisation/MeanDependenceFromSigma_20160.pdf}
        \addFigure{0.48}{../../higgsmassmeasurement/AN-19-248/Figures/Categorisation/MeanDependenceFromSigma_20165.pdf}
        \addFigure{0.48}{../../higgsmassmeasurement/AN-19-248/Figures/Categorisation/MeanDependenceFromSigma_2017.pdf}
        \addFigure{0.48}{../../higgsmassmeasurement/AN-19-248/Figures/Categorisation/MeanDependenceFromSigma_2018.pdf}
    \captionof{figure}
        [DSCB mean values as a function of mass resolution for the four different final states]
        {DSCB mean values as a function of mass resolution for the four different final states: \fourmu (black),
        \foure (red), \twoetwomu (green) and \twomutwoe (blue). Each bin on the x axis stands for a bin 
        in categorization (moving from left to right, the relative mass error comes bigger). 
        \;A) 2016 pre-VFP.
        \;B) 2016 post-VFP.
        \;C) 2017 pre-VFP.
        \;D) 2018 pre-VFP.}
    \label{MeanDependence}
\end{multiFigure}
Finally, examples of the fits for normalisation can be observed in 
Fig.~\ref{signal_normalization_20160} (2016 pre-VFP ggH),
Fig.~\ref{signal_normalization_20165} (2016 post-VFP ggH), 
\ref{signal_normalization_2017} (2017 ggH) and 
\ref{signal_normalization_2018} (2018 ggH)
\begin{multiFigure}
    \centering
        \includegraphics[width=0.96\textwidth]{../../higgsmassmeasurement/AN-19-248/Figures/Categorisation/Yield_ggF_20160.pdf}
    \captionof{figure}
        [Normalization fit for ggH 2016 pre-VFP]
        {Normalization fit for ggH 2016 pre-VFP, for different decay channels, as a function of mass, for the 9 bins of
        % $D'^{\text{VXBS}}_{m_{4\ell}}$. % TODO
        }
    \label{signal_normalization_20160}
\end{multiFigure}
\begin{multiFigure}
    \centering
        \includegraphics[width=0.96\textwidth]{../../higgsmassmeasurement/AN-19-248/Figures/Categorisation/Yield_ggF_20165.pdf}
    \captionof{figure}
        [Normalization fit for ggH 2016 post-VFP]
        {Normalization fit for ggH 2016 post-VFP, for different decay channels, as a function of mass, for the 9 bins of $D'^{\text{VXBS}}_{m_{4\ell}}$.}
    \label{signal_normalization_20165}
\end{multiFigure}
\begin{multiFigure}
    \centering
		\includegraphics[width=0.96\textwidth]{../../higgsmassmeasurement/AN-19-248/Figures/Categorisation/Yield_ggF_2017.pdf}
    \captionof{figure}
        [Normalization fit for ggH 2017]
        {Normalization fit for ggH 2017, for different decay channels, as a function of mass, for the 9 bins of $D'^{\text{VXBS}}_{m_{4\ell}}$.}
    \label{signal_normalization_2017}
\end{multiFigure}
\begin{multiFigure}
    \centering
		\includegraphics[width=0.96\textwidth]{../../higgsmassmeasurement/AN-19-248/Figures/Categorisation/Yield_ggF_2018.pdf}
    \captionof{figure}
        [Normalization fit for ggH 2018]
        {Normalization fit for ggH 2018, for different decay channels, as a function of mass, for the 9 bins of $D'^{\text{VXBS}}_{m_{4\ell}}$.}
    \label{signal_normalization_2018}
\end{multiFigure}

\subsubsection{Result using categorization in 1D model}
% TODO: Include the below?
%The mass error uncertainty evaluated in \ref{sec:EBE} is combined with the four-lepton mass to built
%a two-dimentional likelihood function, $\mathcal{L}$($m_{4\ell}$, \massUnc$|m_{H}$), where again 
%$m_{H}$ is fixed to the value of 125 GeV. \\
%\tablename~\ref{table:2D_model_result_year} shows inclusive results compared with 1D result
%for each year.
The expected $\mH$ measurement uncertainty, in case of no-bkg and no-syst,
implementing the categorization,
is reported in \cref{table:2D_model_result} split for different final state and in \cref{table:2D_model_result_year} for different years.
\begin{table}[ht]	
\begin{center}
    \topcaption
        [Expected Higgs boson mass uncertainty measured implementing the categorization,
        for different final states]
        {Expected Higgs boson mass uncertainty measured implementing the categorization,
        for different final state.
        All mass values are given in \MeV.  
        Statistical only results are considered at this stage of the analysis.
        }
    \begin{tabular}{ccccccc}
    \hline			
    Expected uncertainty	&	\fourmu	&	\foure	&	\twoetwomu	&\twomutwoe	& inclusive & Rel. Improvement \\
    \hline			
        N-1$D'_\text{VXBS}$ (No bkg)	&	124	&	319	&	203	&	236	&	92	&	-6\%	\\
        1$D'_\text{VXBS}$ (No bkg)	&	129	&	357	&	223	&	255	&	99	&	-8\%	\\
    %	N-1$D'_\text{VXBS}$ (No bkg)	&	128	&	371	&	227	&	253	&	98	&	-7\%	\\
    %	1$D'_\text{VXBS}$ (No bkg)	&	134	&	424	&	250	&	279	&	105	&	-	\\
    \hline
    %relative improvement	&	-	&	-	&	-	&	-	&	-	\\
    %\hline
    \end{tabular}
\label{table:2D_model_result}
\end{center}
\end{table}
\begin{table}[ht]	
\begin{center}
    \caption
        [Expected Higgs boson mass uncertainty measured implementing the categorization,
        for different years]
        {Expected Higgs boson mass uncertainty measured implementing the categorization,
        for different years.
        All mass values are given in \MeV.  
        Statistical only results are considered at this stage of the analysis.
        }
    \begin{tabular}{ccccc} % TODO: Add total uncertainty in final column on right-hand side?
        \hline			
    Expected uncertainty	&	2016 pre-VFP	&	2016 post-VFP	&	2017	&	2018	\\
        \hline			
        N-1$D'_\text{VXBS}$ (no-bkg)	&	247	&	255	&	170	&	140	\\
        1$D'_\text{VXBS}$ (No bkg)	&	266	&	276	&	182	&	148	\\
    %	N-1$D'_\text{VXBS}$ (no-bkg)	&	192	&	180	&	147	\\
    %	1$D'_\text{VXBS}$ (No bkg)	&	206	&	194	&	157	\\
    \hline
    \end{tabular}
    \label{table:2D_model_result_year}
\end{center}
\end{table}

