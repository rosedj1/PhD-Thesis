\section{Background Estimation}
\label{sec:bkg_estim_dilep}
The same background processes that were considered in Sec.~\ref{sec:bkg_estim} were also analyzed in this analysis, plus these additional processes:
\begin{itemize}
    \item SM decays of \htofourl are considered as irreducible background (IB)
    \item the rare backgrounds $\ttbar + \PZ$ and triboson production ($\PV\PV\PV$) are included as IB processes
\end{itemize}

The OS Method (Sec.~\ref{sec:os_method}) is used to estimate how many nonsignal events contribute to the four-prompt-lepton signal region (SR).
Events are split into 2 control regions (CRs) that are mutually orthogonal and orthogonal to the SR.
The 2P2F (3P1F) CR is built from those events which have 2 (3) leptons passing tight selection and 2 (1) failing leptons.
These CRs help model those reducible background events with 2 (3) prompt leptons, whose nonprompt leptons pass tight selection when ideally they should not.
By showing good agreement between data and simulated events in the 2P2F and 3P1F CRs, then the RB estimate to the SR becomes more confident.

The misidentification rates $f_{\Pmu}$ $(f_{\Pe})$ for muons (electrons) are estimated using \ztolplm events.
The two leptons forming the \PZ candidate must pass tight selection but a third lepton is required to pass loose selection criteria.
This third lepton definitely does not originate from the \PZ vertex (\ie it is not \emph{prompt}) and so the frequency of this third lepton passing tight selection---the misidentification rate---is calculated.
Low-mass resonances are eliminated by requiring $\mass{2\ell} > 4\GeV$.
Finally, in order to ensure that the lepton pair comes from a \PZ boson instead of photon conversions, the 2 leptons forming the \PZ candidate must satisfy $\abs{\mass{\lplm} - \mass{\PZ_{\text{PDG}}}} < 7\GeV$.
Plots of the measured misidentification rates are shown in \cref{fig:fakerates}.
The full event selection is described in Ref.~\cite{CMS-PAS-HIG-19-007}.
\begin{multiFigure}
    \centering
        \addFigure{0.48}{figures/dilep_res/fakerates_dilep_mass_elec_2016.jpeg}
        \addFigure{0.48}{figures/dilep_res/fakerates_dilep_mass_muon_2016.jpeg}
        \addFigure{0.48}{figures/dilep_res/fakerates_dilep_mass_elec_2017.jpeg}
        \addFigure{0.48}{figures/dilep_res/fakerates_dilep_mass_muon_2017.jpeg}
        \addFigure{0.48}{figures/dilep_res/fakerates_dilep_mass_elec_2018.jpeg}
        \addFigure{0.48}{figures/dilep_res/fakerates_dilep_mass_muon_2018.jpeg}
    \captionof{figure}
        [Misidentification rates of leptons in $\ztolplm + \text{L}$ events using Run 2 data]
        {Misidentification rates of leptons in $\ztolplm + \text{L}$ events using Run 2 data.
        The methodology used to extract the misidentification rates is the same as that described in Sec.~\ref{sec:fr_evtsel}.
        \;A) Electrons in 2016 data.
        \;B) Muons in 2016 data.
        \;C) Electrons in 2017 data.
        \;D) Muons in 2017 data.
        \;E) Electrons in 2018 data.
        \;F) Muons in 2018 data.}
    \label{fig:fakerates}
\end{multiFigure}