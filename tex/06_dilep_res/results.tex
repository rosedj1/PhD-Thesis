\section{Results}
\label{sec:results_dilep}
An analysis of the \mZtwo spectrum is performed to look for any possible low-mass dilepton resonances.
In the case of \htozdzd, in which both of the daughter particles are identical, then a peak in the \mZtwo spectrum is expected at $(\mZone + \mZtwo) / 2$.

A simple counting experiment is performed in many bins across the \mZtwo spectrum.
Using events selected from the \zzd event selection, 353 mass hypotheses \mass{i} are considered for \mZtwo.
The idea is to scan over the entire \mZtwo range $(4.20\text{--}34.98 \GeV)$ in very fine \mZtwo bins, while avoiding the $\Upsilon$ $\bbbar$ bound states.  % TODO: Italicize Upsilon!
To achieve the desired bin width fineness, each subsequent mass hypothesis is increased by 0.5\% of its previous value.
Thus, the mass hypotheses are given by:
\begin{align*}
    \mass{i} = 4.20 \times 1.005^{i} \GeV,
    \text{ where } i = 0,1,2, \ldots, 129, 202, 203, 204, \ldots, 425.
\end{align*}
The bin width is chosen to be two times the \mZtwo resolution.
Concretely, the bin width is equal to 0.04 $(0.10) \times \mass{i}$ for the \fourmu and \twoetwomu (\foure and \twomutwoe) final states.

For each $\mass{i}$, an overall likelihood model $(\lhoodm)$ is defined as:
\begin{equation*}
    \lhoodm =
    \lhoodSR %\left( n_\text{tot}^\text{SR} \right)
    \lhoodsb, %\left( n_\text{tot}^\text{sb} \right)
\end{equation*}
where \lhoodSR is the likelihood that the parameters of interest $\left( \theta_k \right)$ describe the number of events $\left( n^\text{SR}_{\mass{i}, \ell} \right)$ found inside the signal region (SR) for this \mass{i} in a given final state $(\ell)$,
and similarly, \lhoodsb is the likelihood that the same parameters describe the number of events $\left( n^\text{sb}_{\mass{i}, \ell} \right)$ found inside the sidebands (sb)---\ie outside the SR---for this \mass{i} in a given final state $(\ell)$.

Both likelihoods for a given \mass{i} are themselves products of Poisson probabilities\footnote
{
    If the number of expected events (on average) is $\lambda$, then the probability to observe $x$ events is given by the Poisson distribution:
    $\poisson{x \middlepipe \lambda} = \frac{e^{-\lambda} \lambda^x}{x!}$
},
which are defined as:
\begin{align*}
    \lhoodSR &= \prod_\ell{
        \poisson{
            n^\text{SR}_{\mass{i},\ell}
            \middlepipe
            \mu n_{s, \mass{i}, \ell} \rho_{s, \mass{i}, \ell} + \mu_\PH n_{\PH, \mass{i}, \ell} + 
            \sum_b{
                n_{b, \mass{i}, \ell} \rho_{b, \mass{i}, \ell}
                }
            }
    }
    \\
    \text{and} &
    \\
    \lhoodsb &= \prod_\ell{
        \poisson{
            n^\text{sb}_{\ell}
            \middlepipe
            \mu_\PH n_{\PH, \ell} + 
            \sum_b{
                n_{b, \ell} \rho_{b, \ell}
                }
            }
    },
\end{align*}
where $\mu$ is the signal strength parameter, 

$n_\ell$ is the number of 

CL$_\text{s}$~\cite{cowan_asymptotic_2011}.

\begin{multiFigure}
        \centering
        \addFigure{0.48}{figures/dilep_res/yield_hzzd_mZ2_2mu.pdf}
        \addFigure{0.48}{figures/dilep_res/yield_hzzd_mZ2_2e.pdf}
        \captionof{figure}
            [test]
            {testing more}
        \label{fig:yield_hzzd_mZ2}
\end{multiFigure}

\begin{multiFigure}
        \centering
        \addFigure{0.48}{figures/dilep_res/yield_hzdzd_mZ12_4mu.pdf}
        \addFigure{0.48}{figures/dilep_res/yield_hzdzd_mZ12_4e.pdf}
        \addFigure{0.48}{figures/dilep_res/yield_hzdzd_mZ12_2e2mu.pdf}
        % \captionof{figure}
            % []
            % {}
        \label{fig:yield_hzdzd_mZ12}
\end{multiFigure}

\begin{multiFigure}
        \centering
        \addFigure{0.48}{figures/dilep_res/exclus_lim_BR_hzzd_2mu.pdf}
        \addFigure{0.48}{figures/dilep_res/exclus_lim_BR_hzzd_2e.pdf}
        \addFigure{0.48}{figures/dilep_res/exclus_lim_BR_hzzd_2mu_or_2e.pdf}
        \captionof{figure}
            []
            {}
        \label{fig:exclus_lim_BR_hzzd}
\end{multiFigure}

\begin{multiFigure}
        \centering
        \addFigure{0.48}{figures/dilep_res/exclus_lim_BR_hzdzd_2mu.pdf}
        \addFigure{0.48}{figures/dilep_res/exclus_lim_BR_hzdzd_2e.pdf}
        \addFigure{0.48}{figures/dilep_res/exclus_lim_BR_hzdzd_2mu_or_2e.pdf}
        % \captionof{figure}
            % []
            % {}
        \label{fig:exclus_lim_BR_hzdzd}
\end{multiFigure}

\begin{multiFigure}
        \centering
        \addFigure{0.48}{figures/dilep_res/exclus_lim_kappa.pdf}
        % \captionof{figure}
            % []
            % {}
        \label{fig:exclus_lim_kappa}
\end{multiFigure}

