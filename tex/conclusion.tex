\chapter{CONCLUSION}
\label{ch:conclusion}
% The importance of measuring the mass of the Higgs boson 
This dissertation outlines two analyses performed using \pp collision data collected by the CMS detector during the LHC Run 2 period (2016--2018), corresponding to an integrated luminosity of \lumiruntwo.
The first analysis details the steps required to measure the mass of the Higgs boson (\mH) in the \hzzfourl decay channel, where ($\ell = \Pe, \Pmu$).
The measurement of \mH is expected to be $\mH = 125.38 \pm 0.11 \left[\pm 0.11 \stat \pm 0.02\syst \right] \GeV$.  % tot=107 MeV, stat.=105 MeV, syst.=21 MeV.
The analysis improved upon previous measurements of \mH by constraining muon tracks to a vertex that is compatible with the beamspot (the so-called \emph{vertex$+$beamspot constraint}).
The second analysis considers the decay of a SM Higgs boson to BSM dilepton mass resonances, specifically a dark photon $\left( \PZD \right)$, within the context of the Hidden Abelian Higgs Model considering the decay channel $\htozzd \left( \zdzd \right) \to \fourl$.
Upper limits on the branching ratios of $\brof{\htozzd}$, $\brof{\htozdzd}$, and $\brof{\zdtoeepmormumupm}$ are set at the 95\% confidence level as well as the 
Finally, upper limits on the Higgs-mixing parameter $(\kappa)$ are set at the 95\% confidence level.
No significant deviations from the predictions made by the SM are observed.