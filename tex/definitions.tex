% Consider using CMS's standard macros and particle name macros (PENNAMES)
% provided by the ptdr-definitions.sty and heppennames2.sty files, respectively.
% Here's a PDF showing the commands and the output:
% https://twiki.cern.ch/twiki/pub/CMS/Internal/PubGuidelines/macros.pdf

%===============================%
%=== Notes on using \xspace. ===%
%===============================%
% The TeXperts (by the way, if that's not what they're called then something's wrong!)
% recommend NOT using \xspace, since once there are some cases when an extra space will be added:

% \newcommand{\words}{words\xspace}
% (\words)    % Returns: (words)
% [\words]    % Returns: [words ]
% \{\words\}  % Returns: {words }

% The author(s) of xspace didn't included ] and \} in the list of exceptions.
% If you insist on using \xspace, then simply add this to the document preamble:
% \usepackage{xspace}
% \xspaceaddexceptions{]\}}

% So what do the TeXperts recommend instead?
% They say to add `{}` to the end of every macro call:

% \newcommand{\neat}{neat}
% Some \neat words        % Returns: Some neatwords
% Some \neat{} words      % Returns: Some neat words
% Some \neat{}{}{} words  % Returns: Some neat words

% The rule in TeX is simple: after a command name that uses letters, white space is ignored.

%=================================================%
%=== Difference between the kern and the skip. ===%
%=================================================%
% In LaTeX, `\,` (backslash comma) is a thin unbreakable space, also called a 'kern'.
% It has a width of 1/6em (quite literally 1/6th the width of the 'M' in whatever font).
% On the other hand, the `~` (tilde) is a normal unbreakable space, so a little bit wider, also called a 'skip'.
% A skip is the same width of the space between words (the interword space).
% Neither kerns nor skips will be broken across lines, but ~ may get stretched: `5~Kg` -> `5    Kg`.

% Some shorthand
% turn off italics
\newcommand{\etal}{\mbox{et al.}\xspace} %et al. - no preceding comma
\newcommand{\ie}{\mbox{i.e.,}\xspace}     %i.e.
\newcommand{\eg}{\mbox{e.g.,}\xspace}     %e.g.
\newcommand{\etc}{\mbox{etc.}\xspace}     %etc.
\newcommand{\vs}{\mbox{vs.}\xspace}      %vs. No slant according to HIG-16-041 and HIG-19-001.
\newcommand {\mdash}{\ensuremath{\text{---}}} % for use within formulas
\providecommand {\NA}{\ensuremath{\text{---}}}    % for Not applicable (or available). Needs to be renewcommanded for APS to \cdots
\providecommand{\middlepipe}{\ensuremath{\; \middle| \;}\xspace{}}  % The spacing to the left and right of \middle is weird, so manually fix it.

% \providecommand{\paren}[1]{\ensuremath{\left( \text{#1} \right)}\xspace}  % Ends up spacing out the internal text too much.
% some terms whose definition we may change
\newcommand {\Lone}{Level-1\xspace} % Level-1 or L1 ?
\newcommand {\Ltwo}{Level-2\xspace}
\newcommand {\Lthree}{Level-3\xspace}

% Units.
% Xunwu used `\,` (a kern) at the beginning of each unit.
% Jake, consider getting rid of \mbox below!
\newcommand{\de}{\ensuremath{^\circ}}
\newcommand{\tentothe}[1]{\ensuremath{\times \text{10}^\text{#1}}}
\newcommand{\tentotheminus}[1]{\ensuremath{\times \text{10}^\text{$-$#1}}}
\newcommand{\unit}[1]{\ensuremath{\text{\,#1}}\xspace}
% Base units.
\newcommand{\tesla}{\ensuremath{\,\text{T}}\xspace}
\newcommand{\gram}{\ensuremath{\,\text{g}}\xspace}
\newcommand{\meter}{\ensuremath{\,\text{m}}\xspace}
\newcommand{\snd}{\ensuremath{\,\text{s}}\xspace}
% Mass.
\newcommand{\Kg}{\ensuremath{\,\text{Kg}}\xspace}
\newcommand{\tonne}{\ensuremath{\,\text{t}}\xspace}
% Volume.
\newcommand{\mL}{\ensuremath{\,\text{mL}}\xspace}
% Temperature.
\newcommand{\kelvin}{\ensuremath{\,\text{K}}\xspace}
% Length.
\newcommand{\Km}{\ensuremath{\,\text{Km}}\xspace}
\newcommand{\cm}{\ensuremath{\,\text{cm}}\xspace}
\newcommand{\mm}{\ensuremath{\,\text{mm}}\xspace}
\newcommand{\mum}{\ensuremath{\,\mu\text{m}}\xspace}
\newcommand{\micron}{\mum}
\newcommand{\nm}{\ensuremath{\,\text{nm}}\xspace}
% Time.
\newcommand{\ms}{\ensuremath{\,\text{ms}}\xspace}
\newcommand{\mus}{\ensuremath{\,\mu\text{s}}\xspace}
\newcommand{\ns}{\ensuremath{\,\text{ns}}\xspace}
% Energy.
\newcommand{\KJ}{\ensuremath{\,\text{KJ}}\xspace}
\newcommand{\GJ}{\ensuremath{\,\text{GJ}}\xspace}
\newcommand{\KeV}{\ensuremath{\,\text{Ke\hspace{-.08em}V}}\xspace}
\newcommand{\MeV}{\ensuremath{\,\text{Me\hspace{-.08em}V}}\xspace}
\newcommand{\MeVns}{\ensuremath{\text{Me\hspace{-.08em}V}}\xspace} % no leading thinspace
\newcommand{\GeV}{\ensuremath{\,\text{Ge\hspace{-.08em}V}}\xspace}
\newcommand{\GeVns}{\ensuremath{\text{Ge\hspace{-.08em}V}}\xspace} % no leading thinspace
\newcommand{\gev}{\GeV}
\newcommand{\TeV}{\ensuremath{\,\text{Te\hspace{-.08em}V}}\xspace}
\newcommand{\TeVns}{\ensuremath{\text{Te\hspace{-.08em}V}}\xspace} % no leading thinspace
\newcommand{\PeV}{\ensuremath{\,\text{Pe\hspace{-.08em}V}}\xspace}
\newcommand{\KeVc}{\ensuremath{{\,\text{Ke\hspace{-.08em}V\hspace{-0.16em}/\hspace{-0.08em}}c}}\xspace}
\newcommand{\MeVc}{\ensuremath{{\,\text{Me\hspace{-.08em}V\hspace{-0.16em}/\hspace{-0.08em}}c}}\xspace}
\newcommand{\GeVc}{\ensuremath{{\,\text{Ge\hspace{-.08em}V\hspace{-0.16em}/\hspace{-0.08em}}c}}\xspace}
\newcommand{\GeVcns}{\ensuremath{{\text{Ge\hspace{-.08em}V\hspace{-0.16em}/\hspace{-0.08em}}c}}\xspace} % no leading thinspace
\newcommand{\TeVc}{\ensuremath{{\,\text{Te\hspace{-.08em}V\hspace{-0.16em}/\hspace{-0.08em}}c}}\xspace}
\newcommand{\KeVcc}{\ensuremath{{\,\text{Ke\hspace{-.08em}V\hspace{-0.16em}/\hspace{-0.08em}}c^\text{2}}}\xspace}
\newcommand{\MeVcc}{\ensuremath{{\,\text{Me\hspace{-.08em}V\hspace{-0.16em}/\hspace{-0.08em}}c^\text{2}}}\xspace}
\newcommand{\GeVcc}{\ensuremath{{\,\text{Ge\hspace{-.08em}V\hspace{-0.16em}/\hspace{-0.08em}}c^\text{2}}}\xspace}
\newcommand{\GeVccns}{\ensuremath{{\text{Ge\hspace{-.08em}V\hspace{-0.16em}/\hspace{-0.08em}}c^\text{2}}}\xspace} % no leading thinspace
\newcommand{\TeVcc}{\ensuremath{{\,\text{Te\hspace{-.08em}V\hspace{-0.16em}/\hspace{-0.08em}}c^\text{2}}}\xspace}
% Cross section.
\newcommand{\mb}{\ensuremath{\,\text{mb}}\xspace}
\newcommand{\pb}{\ensuremath{\,\text{pb}}\xspace}
\newcommand{\fb}{\ensuremath{\,\text{fb}}\xspace}
% Luminosity.
%--- Don't use the \mbox commands. ---%
% \newcommand{\abinv} {\mbox{\ensuremath{\,\text{ab}^{-1}}}\xspace}
% \newcommand{\fbinv} {\mbox{\ensuremath{\,\text{fb}^{-1}}}\xspace}
\newcommand{\fbinv}{\ensuremath{\,\text{fb}^{\text{$-$}1}}\xspace}
% \newcommand{\invfb}{\ensuremath{\mbox{fb}^{\scriptscriptstyle -1}}\xspace}  % Xunwu's usage.
% \newcommand{\pbinv} {\mbox{\ensuremath{\,\text{pb}^{-1}}}\xspace}
% \newcommand{\nbinv} {\mbox{\ensuremath{\,\text{nb}^{-1}}}\xspace}
\newcommand{\mubinv} {\ensuremath{\,\mu\text{b}^{-1}}\xspace}
\newcommand{\mbinv} {\ensuremath{\,\text{mb}^{-1}}\xspace}
\newcommand{\percms}{\ensuremath{\,\text{cm}^{-2}\,\text{s}^{-1}}\xspace}
\newcommand{\lumiint}{\ensuremath{\lumi_\text{int}}\xspace}
\newcommand{\lumi}{\ensuremath{\mathcal{L}}\xspace}
\newcommand{\Lumi}{\lumi}
% Luminosity values.
\newcommand{\LvLow}  {\ensuremath{\lumi = \text{10}^\text{32}\,\text{cm}^\text{$-$2}\,\text{s}^\text{$-$1}}\xspace}
\newcommand{\LLow}   {\ensuremath{\lumi = \text{10}^\text{33}\,\text{cm}^\text{$-$2}\,\text{s}^\text{$-$1}}\xspace}
\newcommand{\lowlumi}{\ensuremath{\lumi = \text{2}\times \text{10}^\text{33}\,\text{cm}^\text{$-$2}\,\text{s}^\text{$-$1}}\xspace}
\newcommand{\LMed}   {\ensuremath{\lumi = \text{2}\times \text{10}^\text{33}\,\text{cm}^\text{$-$2}\,\text{s}^\text{$-$1}}\xspace}
\newcommand{\LHigh}  {\ensuremath{\lumi = \text{10}^\text{34}\,\text{cm}^\text{$-$2}\,\text{s}^\text{$-$1}}\xspace}
\newcommand{\hilumi} {\ensuremath{\lumi = \text{10}^\text{34}\,\text{cm}^\text{$-$2}\,\text{s}^\text{$-$1}}\xspace}
\newcommand{\lum}{\ensuremath{\,\text{(lumi)}}\xspace}
\newcommand{\lumisixteen}{\ensuremath{35.9\fbinv}\xspace}
\newcommand{\lumiruntwo}{\ensuremath{137.1\fbinv}\xspace}

% analysis tools
\newcommand{\GEANTfour} {{\textsc{Geant4}}\xspace}  % \textsc is "small caps".
\newcommand{\HERWIG} {{\textsc{herwig}}\xspace}
\newcommand{\HERWIGpp} {{\textsc{herwig++}}\xspace}
\newcommand{\HERWIGSeven}{{\textsc{HERWIG7}}\xspace}
\newcommand{\MADGRAPH} {\textsc{MadGraph}\xspace}
\newcommand{\MCATNLO} {\textsc{mc@nlo}\xspace}
\newcommand{\MCFM} {{\textsc{mcfm}}\xspace} % \newcommand{\MCFM} {\textsc{mcfm}\xspace}
\newcommand{\POWHEG} {{\textsc{powheg}}\xspace}
\newcommand{\PYTHIA} {{\textsc{pythia}}\xspace}
\newcommand{\SHERPA} {{\textsc{sherpa}}\xspace}
\newcommand{\TOPpp} {{\textsc{TOP++}}\xspace}
\newcommand{\MGvATNLO} {\MADGRAPH{}5\_a\MCATNLO}
\newcommand{\FEWZ} {{\textsc{fewz}}\xspace}

%=== Math ===%
% absolute value
\providecommand{\abs}[1]{\ensuremath{\lvert #1 \rvert}}
% Calculus: Roman face derivative.
\newcommand{\dd}[2]{\ensuremath{\frac{\cmsSymbolFace{d} #1}{\cmsSymbolFace{d} #2}}}
\newcommand{\ddinline}[2]{\ensuremath{\cmsSymbolFace{d} #1/\cmsSymbolFace{d} #2}}
\newcommand{\rd}{\ensuremath{\cmsSymbolFace{d}}}
\newcommand{\re}{\ensuremath{\cmsSymbolFace{e}}}
% Statistics.
\newcommand{\CL}{\ensuremath{\text{CL}}\xspace} % needs to be overridden to C.L. for APS. Look out for \CL.
\newcommand{\CLs}{\ensuremath{\text{CL}_\text{s}}\xspace}
\newcommand{\CLsb}{\ensuremath{\text{CL}_\text{s+b}}\xspace}
\newcommand{\stat}{\ensuremath{\,\text{(stat)}}\xspace}
\newcommand{\syst}{\ensuremath{\,\text{(syst)}}\xspace}
\newcommand{\theo}{\ensuremath{\,\text{(theo)}}\xspace}
\newcommand{\SoB}{\ensuremath{S/B}\xspace}
%  Experiments
\newcommand{\DZERO}{D0\xspace}     %etc.

% Physics symbols.
\newcommand{\abseta}{\ensuremath{\abs{\eta}}\xspace}
\newcommand{\PT}{\ensuremath{p_{\mathrm{T}}}\xspace}
\newcommand{\pt}{\PT}
\newcommand{\pT}{\PT}
\newcommand{\ET}{\ensuremath{E_{\mathrm{T}}}\xspace}
\newcommand{\et}{\ET}
\newcommand{\MT}{\ensuremath{M_{\mathrm{T}}}\xspace}
\newcommand{\mT}{\MT}
\newcommand{\mTii}{\ensuremath{m_{\mathrm{T2}}}\xspace}
\newcommand{\Em}{\ensuremath{E\hspace{-0.6em}/}\xspace}
\newcommand{\Pm}{\ensuremath{p\hspace{-0.5em}/}\xspace}
\newcommand{\PTm}{\ensuremath{{p}_\mathrm{T}\hspace{-1.02em}/\kern 0.5em}\xspace}
\newcommand{\PTslash}{\PTm}
\newcommand{\kt}{\ensuremath{k_{\mathrm{T}}}\xspace}
\newcommand{\ETm}{\ensuremath{E_{\mathrm{T}}^{\text{miss}}}\xspace}
\newcommand{\MET}{\ETm}
\newcommand{\ETmiss}{\ETm}
\newcommand{\ptmiss}{\ensuremath{\pt^\text{miss}}\xspace}
\newcommand{\ETslash}{\ensuremath{E_{\mathrm{T}}\hspace{-1.1em}/\kern0.45em}\xspace}
\newcommand{\VEtmiss}{\ensuremath{{\vec E}_{\mathrm{T}}^{\text{miss}}}\xspace}
\newcommand{\ptvec}{\ensuremath{{\vec p}_{\mathrm{T}}}\xspace}
\newcommand{\ptvecmiss}{\ensuremath{{\vec p}_{\mathrm{T}}^{\kern1pt\text{miss}}}\xspace}
\newcommand{\tauh}{\ensuremath{\PGt_\mathrm{h}}\xspace}
\newcommand{\sqrtsthirteen}{\ensuremath{\sqrt{s} = 13 \TeV}\xspace}
\newcommand{\sqrtsNN}{\ensuremath{\sqrt{\smash[b]{s_{_{\mathrm{NN}}}}}}\xspace}
\newcommand{\MHT}{\ensuremath{H_{\mathrm{T}}^{\text{miss}}}\xspace}
\newcommand{\mht}{\MHT}
\newcommand{\htvecmiss}{\ensuremath{\vec{H}_{\text{T}}^{\text{miss}}}\xspace}
\newcommand{\wangle}{\ensuremath{\sin^{2}\theta_{\text{eff}}^\text{lept}(M^2_{\PZ})}\xspace}
\newcommand{\alpS}{\ensuremath{\alpha_S}\xspace}
% Mass commands.
\providecommand{\mass}[1]{\ensuremath{m_{#1}}\xspace}
\providecommand{\mll}{\ensuremath{m_{\ell\ell}}\xspace}

%=== Higgs mass symbols ===%
\providecommand{\mh}{\ensuremath{m_{\PH}}\xspace}
\providecommand{\mH}{\mh}
\providecommand{\mfourl}{\ensuremath{m_{\fourl}}\xspace}
\providecommand{\mfourlerr}{\ensuremath{\sigma_{\mfourl}}\xspace}
\providecommand{\relmfourlerr}{\ensuremath{\frac{\mfourlerr}{\mfourl}}\xspace}
\providecommand{\Dkinbkg}{\ensuremath{\mathcal{D}^{\text{kin}}_{\text{bkg}}}\xspace}



%=== Particles ===%
\newcommand{\Pn}{\ensuremath{\mathrm{n}}}
\newcommand{\Pp}{\ensuremath{\mathrm{p}}}
\newcommand{\PV}{\ensuremath{\mathrm{V}}\xspace}
\newcommand{\PW}{\ensuremath{\mathrm{W}}\xspace}
\newcommand{\PWpm}{\ensuremath{\mathrm{W}^\pm}\xspace}
\newcommand{\PX}{\ensuremath{\mathrm{X}}\xspace}
\newcommand{\PZ}{\ensuremath{\mathrm{Z}}\xspace}
\newcommand{\PH}{\ensuremath{\mathrm{H}}\xspace}
\newcommand{\Pe}{\ensuremath{\mathrm{e}}\xspace}
\newcommand{\Pg}{\ensuremath{\mathrm{g}}\xspace}
\newcommand{\Pq}{\ensuremath{\mathrm{q}}\xspace}
\newcommand{\Paq}{\ensuremath{\overline{\Pq}}\xspace}
\newcommand{\Pqb}{\ensuremath{\mathrm{b}}\xspace}
\newcommand{\Paqb}{\ensuremath{\overline{\Pqb}}\xspace}
\newcommand{\Pqt}{\ensuremath{\mathrm{t}}\xspace}
\newcommand{\Paqt}{\ensuremath{\overline{\Pqt}}\xspace}
\newcommand{\PGn}{\ensuremath{\nu}\xspace} % generic neutrino
\newcommand{\PAGn}{\ensuremath{\overline{\nu}}\xspace} % generic neutrino
\newcommand{\pname}{\text{P. Roton}\xspace}
% Modified.
\newcommand{\Zone}{\ensuremath{\PZ_{\mathrm{1}}}\xspace}
\newcommand{\Ztwo}{\ensuremath{\PZ_{\mathrm{2}}}\xspace}

%=== Particle combos. ===%
\newcommand{\pp}{\ensuremath{\Pp \Pp}\xspace}
\newcommand{\ellpellm}{\ensuremath{\ell^+ \ell^-}\xspace}
\newcommand{\ee}{\ensuremath{\Pe^+ \Pe^-}\xspace}
\newcommand{\fourl}{\ensuremath{4\ell}\xspace}
\newcommand{\fourmu}{\ensuremath{4\mu}\xspace}
\newcommand{\foure}{\ensuremath{4\Pe}\xspace}
\newcommand{\twoetwomu}{\ensuremath{2\Pe2\mu}\xspace}
\newcommand{\twomutwoe}{\ensuremath{2\mu2\Pe}\xspace}
\newcommand{\ZX}{\ensuremath{\PZ \PX}\xspace}
\newcommand{\XX}{\ensuremath{\PX \PX}\xspace}
\newcommand{\htwo}{\ensuremath{\text{H}_2}\xspace}
% Xunwu's.
\providecommand{\ttH}{\ensuremath{\Pqt\Paqt\PH}\xspace}
\providecommand{\ggH}{\ensuremath{\Pg\Pg\PH}\xspace}
\providecommand{\qqH}{\mbox{VBF}\xspace}
\providecommand{\ZH}{\ensuremath{\PZ\PH}\xspace}
\providecommand{\WH}{\ensuremath{\PW\PH}\xspace}
\providecommand{\VH}{\ensuremath{\PV\PH}\xspace}
\providecommand{\bbH}{\ensuremath{\Pqb\Paqb\PH}\xspace}
\providecommand{\tHq}{\ensuremath{\Pqt\PH\Pq}\xspace}
\providecommand{\tHW}{\ensuremath{\Pqt\PH\PW}\xspace}
\providecommand{\qqZH}{\ensuremath{\Pq\Pq\PZ\PH}\xspace}
\providecommand{\ggZH}{\ensuremath{\Pg\Pg\PZ\PH}\xspace}
\providecommand{\DY}{\mbox{DY}\xspace}
\providecommand{\ttbar}{\ensuremath{\Pqt\Paqt}\xspace}
\providecommand{\ggZZ}{\ensuremath{\Pg\Pg\PZ\PZ}\xspace}
\providecommand{\WZ}{\ensuremath{\PW\PZ}\xspace}
\providecommand{\ZZ}{\ensuremath{\PZ\PZ}\xspace}
\providecommand{\VVV}{\ensuremath{\PV\PV\PV}\xspace}

\newcommand{\Zee}{\ensuremath{\PZ(\Pe\Pe)}\xspace}
\newcommand{\Zll}{\ensuremath{\PZ(\ell\ell)}\xspace}
\newcommand{\DYjets}{\ensuremath{\PZ/\gamma^\ast\text{+jets}}\xspace}
\newcommand{\tZq}{\ensuremath{\Pqt\PZ\Pq}\xspace}
\newcommand{\tZW}{\ensuremath{\Pqt\PZ\PW}\xspace}

\newcommand{\kappaz}{\ensuremath{\kappa\smash[b]{^{\PGn\PAGn}}}\xspace}
\newcommand{\kappaV}{\ensuremath{\kappa_{\mathrm{V}}}\xspace}
\newcommand{\kappaF}{\ensuremath{\kappa_{\mathrm{F}}}\xspace}
\newcommand{\kappazi}{\ensuremath{\kappa_{i}\smash[b]{^{\PGn\PAGn}}}\xspace}

% Reactions/Processes.
\newcommand{\Ztolplm}{\ensuremath{\PZ \to \ellpellm}\xspace}
\newcommand{\hmm}{\ensuremath{\PH \to \mu\mu}\xspace}
\newcommand{\zmm}{\ensuremath{\PZ \to \mu\mu}\xspace}
\newcommand{\htozz}{\ensuremath{\PH \to \PZ\PZ^{\ast}}\xspace}
\newcommand{\htoyy}{\ensuremath{\PH \to \gamma\gamma}\xspace}
\newcommand{\htofourl}{\ensuremath{\PH \to \fourl}\xspace}
\newcommand{\hzzfourl}{\ensuremath{\htozz \to \fourl}\xspace}
\newcommand{\hzxxxfourl}{\ensuremath{\PH \to \ZX / \XX \to \fourl}\xspace}

%=== RedBkg ===%
\newcommand{\looselep}{\ensuremath{\ell_{\mathrm{L}}}\xspace}
\newcommand{\ZplusL}{\ensuremath{\PZ + \looselep}\xspace}
\newcommand{\ZplusX}{\ensuremath{\PZ + \mathrm{X}}\xspace}
\newcommand{\Zplusjets}{\ensuremath{\PZ + \text{jets}}\xspace}
\newcommand{\WZplusjets}{\ensuremath{\WZ + \text{jets}}\xspace}
\newcommand{\Zgammastar}{\ensuremath{\PZ\gamma^\ast}\xspace}
% \newcommand{\Zorgammastar}{\ensuremath{\PZ/\gamma^\ast}\xspace}
\newcommand{\ttbarplusjets}{\ensuremath{\Pqt\Paqt + \text{jets}}\xspace}
% CRs.
\newcommand{\ntwoPtwoF}{\ensuremath{N_{\text{2P2F}}}\xspace}
\newcommand{\nthreePoneF}{\ensuremath{N_{\text{3P1F}}}\xspace}
\newcommand{\nfourP}{\ensuremath{N_\text{4P}}\xspace}
\newcommand{\nfourPRB}{\ensuremath{\nfourP^\text{RB}}\xspace}
\newcommand{\nthreePoneFzz}{\ensuremath{N_{\text{3P1F}}^{\ZZ}}\xspace}
\newcommand{\xtwoprompt}{\ensuremath{X_{2\text{pr}}}\xspace}
\newcommand{\xthreeprompt}{\ensuremath{X_{3\text{pr}}}\xspace}
\newcommand{\xfourprompt}{\ensuremath{X_{4\text{pr}}}\xspace}
\newcommand{\xfourpromptzz}{\ensuremath{X_{4\text{pr}}^{\ZZ}}\xspace}
% Efficiencies.
\newcommand{\effprpass}{\ensuremath{\epsilon^\text{pr}_\text{P}}\xspace}
\newcommand{\effprfail}{\ensuremath{\epsilon^\text{pr}_\text{F}}\xspace}
\newcommand{\effnppass}{\ensuremath{\epsilon^\text{np}_\text{P}}\xspace}
\newcommand{\effnpfail}{\ensuremath{\epsilon^\text{np}_\text{F}}\xspace}
% Weights.
\newcommand{\wgttwoprompttotwoPtwoF}{\ensuremath{w_{\text{2pr} \to \text{2P2F}}\xspace}}
\newcommand{\wgttwoprompttothreePoneF}{\ensuremath{w_{\text{2pr} \to \text{3P1F}}\xspace}}
\newcommand{\wgttwoprompttofourP}{\ensuremath{w_{\text{2pr} \to \text{4P}}\xspace}}
\newcommand{\wgtthreeprompttothreePoneF}{\ensuremath{w_{\text{3pr} \to \text{3P1F}}\xspace}}
\newcommand{\wgtthreeprompttofourP}{\ensuremath{w_{\text{3pr} \to \text{4P}}\xspace}}
\newcommand{\wgtfourpromptzztothreePoneF}{\ensuremath{w_{\text{4pr},\ZZ \to \text{3P1F}}\xspace}}

\newcommand{\ggzzfourl}{\ensuremath{\mathrm{gg \to \PZ\PZ \to 4\ell}}\xspace}
\newcommand{\ggzzstarfourl}{\ensuremath{\mathrm{gg \to \PZ\PZ^\ast \to 4\ell}}\xspace}
\newcommand{\qqzz}{\ensuremath{\mathrm{q\bar{q} \to \PZ\PZ}}\xspace}
\newcommand{\qqzzfourl}{\ensuremath{\qqzz \to 4\ell}\xspace}
\newcommand{\qqggzzfourl}{\ensuremath{\mathrm{q\bar{q}/ \ggzzfourl}}\xspace}
\newcommand{\gghzzfourl}{\ensuremath{\mathrm{gg \to \hzzfourl}}\xspace}

% To-be sorted.
\newcommand{\br}{\ensuremath{\mathcal{B}}\xspace}
\newcommand{\BRof}[1]{\ensuremath{{\mathcal{B} \left( #1 \right)}}\xspace}
\newcommand{\brhmm}{\ensuremath{{\mathcal{B}(\PH \to \mu\mu)}}\xspace}
\newcommand{\brhff}{\ensuremath{{\mathcal{B}(\PH \to f\overline{f})}}\xspace}
\newcommand{\detajj}{\ensuremath{\Delta\eta_{\mathrm{jj}}}\xspace}
\newcommand{\dphijj}{\ensuremath{\Delta\phi_{\mathrm{jj}}}\xspace}

\newcommand{\NNPDF} {\textsc{nnpdf}\xspace}

\newcommand{\dzeroPV}{\ensuremath{\text{d0}_{\text{PV}}}\xspace}
\newcommand{\dzeroBS}{\ensuremath{\text{d0}_{\text{BS}}}\xspace}

\newcommand{\dptoverptsquare}{\ensuremath{(\pt^{Roch} - \pt^{Gen})/\pt^2}\xspace}

\newcommand{\RochCorr}{\textit{Rochester correction}\xspace}
\newcommand{\FSR}{\textit{FSR recovery}\xspace}
\newcommand{\GeoFit}{\textit{GeoFit correction}\xspace}